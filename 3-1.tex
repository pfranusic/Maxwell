%% 3-1.tex
\section{Vector integrals; the line integral of $\grad\Psi$}

Everything in Chapter 2 can be summarized with one rule:
the operators $\partial/\partial x$, $\partial/\partial y$, and $\partial/\partial z$
are the three components of the vector operator $\grad$.
In this chapter we explain the meanings of the divergence and curl operations,
and we develop three important theorems.

We begin with the fundamental theorem of calculus for one variable.
The function $f(x)$ can be imagined as a curve that hovers above and below the x-axis.
The function $f'(x)$ is the slope of $f(x)$.  That is, $f'(x) = df/dx$.
The fundamental theorem states that, given two points $a$ and $b$,  
the total difference between $f(a)$ and $f(b)$ can be computed by 
adding up all the small differences $df$ between $a$ and $b$,
where each small difference $df$ is the product of a slope $f'(x)=df/dx$ and a small distance $dx$.
\begin{equation*}
  f(b) - f(a) = \int _a ^b f'(x) dx
\end{equation*}

We are given a scalar field $\psi(x,y,z)$.
These are the values $\psi$ of some scalar quantity (such as temperature)
at each point $(x,y,z)$ in some three-dimensional space (such as a block of concrete).
We are also given a vector field $\grad\psi(x,y,z)$.
These are the slopes $(\frac{\partial\psi}{\partial x},
\frac{\partial\psi}{\partial y},\frac{\partial\psi}{\partial z})$
of scalar field $\psi$ at each point $(x,y,z)$.

\newpage
Consider Figure 3-1, where $\Gamma$ is some curve in the scalar field $\psi(x,y,z)$ 
that starts at point $(1)=(x_1,y_1,z_1)$ and ends at point $(2)=(x_2,y_2,z_2)$.
The vector $d\bv{s}= (\scriptstyle{\partial x,\partial y,\partial z})$
is an infinitesimal line element along the curve $\Gamma$.
We may write the fundamental theorem of calculus in vector form.
The total difference between $\psi(1)$ and $\psi(2)$ can be computed by 
adding up all the small differences $\partial\psi$ between $(1)$ and $(2)$,
where each small difference $\partial\psi$ is the scalar product of 
a slope $\grad\psi=(\frac{\partial\psi}{\partial x},
\frac{\partial\psi}{\partial y},\frac{\partial\psi}{\partial z})$
and a small distance $d\bv{s}= (\scriptstyle{\partial x,\partial y,\partial z})$.
%% The lower limit of the integral... oh what a mess. But it works. :-)
%% Eq. (3.1)
\begin{equation}
  \psi(2) - \psi(1) = \int
  _{\hspace{-2ex}\raisebox{-1ex}{\mbox{$\genfrac{}{}{0pt}{1}{(1)}{\mathrm{along\ }\Gamma}$}}}
  ^{(2)} (\grad \psi) \cdot d\bv{s}
\end{equation}

In Figure 3-2, 
$(\grad\psi)_t$ is the component of $\grad\psi$ in the direction of $\Delta \bv{s}_i$.
Simple trigonometry gives us $(\grad\psi)_t = |\grad\psi| \cos \theta$,
where $\theta$ is the angle between $\grad\psi$ and $(\grad\psi)_t$.
$\Delta s_i$ is the magnitude of the vector $\Delta \bv{s}_i$.
I.e., $\Delta s_i = |\Delta \bv{s}_i|$.
So we have $(\grad\psi)_t \Delta s = |\grad\psi| |\Delta \bv{s}_i| \cos \theta$.
Now we apply the identity $|\bv{a}||\bv{b}|\cos\theta=\bv{a}\cdot\bv{b}$ 
and drop the $i$ subscripts to get
%% Eq. (3.2)
\begin{equation}
  (\grad \psi)_t \; \Delta s = (\grad \psi) \cdot \Delta \bv{s}
\end{equation}

The small interval vector $\Delta \bv{R} = (\Delta x, \Delta y, \Delta z)$
was introduced in section 2-3.  We now take equation (2.15), where
$\Delta T = \grad T \cdot \Delta \bv{R}$, and replace terms to get 
\begin{equation*}
  \Delta \psi_1 = (\grad \psi)_1 \cdot \Delta \bv{s}_1
\end{equation*}

In figure 3-2, point $a$ is the first point after (1) along the curve $\Gamma$.
It's obvious that $\Delta \psi_1 = \psi(a) - \psi(1)$.
Combining these two equations for $\Delta \psi_1$ we have
%% Eq. (3.3)
\begin{equation}
  \Delta \psi_1 = \psi(a) - \psi(1) = (\grad \psi)_1 \cdot \Delta \bv{s}_1
\end{equation}

We do the same thing for $\Delta \psi_2$.
%% Eq. (3.4)
\begin{equation}
  \Delta \psi_2 = \psi(b) - \psi(a) = (\grad \psi)_2 \cdot \Delta \bv{s}_2
\end{equation}

We add equations (3.3) and (3.4) to get the sum $\Delta \psi_1 + \Delta \psi_2$.
Notice that the $\psi(a)$ terms cancel out.
%% Eq. (3.5)
\begin{equation}
  \psi(b) - \psi(1) = (\grad \psi)_1 \cdot \Delta \bv{s}_1 + (\grad \psi)_2 \cdot \Delta \bv{s}_2
\end{equation}

If we add up all of the segments along $\Gamma$ we get
%% Eq. (3.6)
\begin{equation}
  \psi(2) - \psi(1) = \sum (\grad \psi)_i \cdot \Delta \bv{s}_i
\end{equation}

It's okay to drop the parentheses around $(\grad\psi)$.
%% Eq. (3.7)
\begin{equation}
  (\grad \psi) \cdot d\bv{s} = \grad \psi \cdot d\bv{s}
\end{equation}

Consider the sum in equation (3.6).
If we let $\Delta \bv{s}_i$ approach 0
we can replace the summation sign $\sum$ with an integral sign $\int$.
This theorem is correct for any curve $\Gamma$ from (1) to (2).

%% Eq. (3.8)
\vspace{2ex} \hspace{2em} \textsc{Fundamental Theorem:}
\begin{equation}
  \psi(2) - \psi(1) = \int
  _{\hspace{-3ex}\raisebox{-1ex}{\mbox{$\genfrac{}{}{0pt}{1}{(1)}{\mathrm{any\ curve}}$}}}
  ^{(2)} \grad \psi \cdot d\bv{s} 
\end{equation}

