%% 4-7.tex
\section{Field of a sphere of charge}

Figure 4-11 shows a small sphere with radius $a$
and filled with a uniform distribution of charge with density $\rho$.
The point $P$ lies outside the surface of the small sphere at a distance $R$.
We imagine that $P$ lies on the surface $S$ of a larger sphere with radius $R$.
This surface $S$ is concentric with the surface of the small sphere.
Now we ask: What is the electric field at $P$?

First we express the flux through the larger sphere.
We assume that $\bv{E}$ is everywhere directed away from the center of the sphere.
Given that $E = |\bv{E}|$, and that the area of surface $S$ is $4\pi R^2$, 
the flux through surface $S$ is simply the product of $E$ and the surface area.
%% nonumber
\begin{equation*}
  \int E_n \; da = E \cdot 4 \pi R^2
\end{equation*}

Gauss' law expresses flux differently.
Gauss' law expresses flux in terms of the total charge $Q$ within surface $S$.
%% nonumber
\begin{equation*}
  \int E_n \; da = \frac{Q}{\epsilon_0}
\end{equation*}

Both equations express the flux of $\bv{E}$.
The first equation expresses the flux \emph{through} $S$.
The second equation expresses the flux \emph{within} $S$.
They are the same thing! So we can make the right sides equal.
%% nonumber
\begin{equation*}
  E \cdot 4 \pi R^2 = \frac{Q}{\epsilon_0}
\end{equation*}

Now we simply divide both sides by $4\pi R^2$ to get an equation for $E$,
the magnitude of $\bv{E}$ at any point outside surface $S$.
(Somehow $R$ is replace by $r$).
%% Eq. (4.39)
\begin{equation}
  E = \frac{1}{4\pi\epsilon_0} \; \frac{Q}{r^2}
\end{equation}

