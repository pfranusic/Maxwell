%% 2-3.tex
\section{Derivatives of fields --- the gradient}

This section introduces the concept of the \emph{gradient}.
It defines $\grad T$ (pronounced ``grad tee''), the gradient of the temperature function,
as a vector whose three components in any Cartesian coordinate system are partial derivatives.
It also demonstrates that vectors are the same regardless of the orgin and rotation of the 
coordinate system.

We start by asking, what is the derivative of a scalar field?
Consider the three partial derivatives 
$\partial T/ \partial x$, $\partial T/ \partial y$, and $\partial T/ \partial z$.
Perhaps these are the components of a vector.
%% Eq. (2.11)
%% '\;' is a small space.
%% '\stackrel{?}{=}' prints a question mark over an equal sign.
\begin{equation}
  \Bigg( \dfdx{T}{x} , \dfdx{T}{y} , \dfdx{T}{y} \Bigg)
  \; \stackrel{?}{=} \; \mathrm{a\ vector}
\end{equation}

We wish to demonstrate that this is indeed a vector.
We will show that when the coordinate system is rotated,
the three derivatives transform among themselves in the correct way.
Consider some vector $\bv{A}$ with components $(A_x,A_y,A_z)$
and three numbers $B_1$, $B_2$, and $B_3$,
which are combined with the components of $\bv{A}$ in the following way
to compute some scalar $S$.
Then by Eq. (2.1) it must be true that the three numbers $B_1$, $B_2$, and $B_3$
are the components $(B_x,B_y,B_z)$ of some vector $\bv{B}$.
%% Eq. (2.12)
\begin{equation}
  A_x B_1  +  A_y B_2  + A_z B_3 = S
\end{equation}

Now consider a solid block of material where the temperature at any point is $T(x,y,z)$
and the change in temperature between any two nearby points is $\Delta T(x,y,z)$.
We use the total differential from Eq. (2.7) to specify $\Delta T$.
%% Eq. (2.13)
\begin{equation}
  \Delta T  =  \dfdx{T}{x} \Delta x  +  \dfdx{T}{y} \Delta y  +  \dfdx{T}{z} \Delta z 
\end{equation}

Note that the right side is in the form of a scalar product of two vectors.
The second vector has the three components $\Delta x, \Delta y, \Delta z$.
The first vector has the three components
$\partial T/ \partial x$, $\partial T/ \partial y$, and $\partial T/ \partial z$.
This first vector has a special notation: $\grad T$.
%% Eq. (2.14)
\begin{equation}
  \mathrm{grad\ } T = \grad T = \Bigg( \dfdx{T}{x} , \dfdx{T}{y} , \dfdx{T}{y} \Bigg)
\end{equation}

Let $\Delta \bv{R} = (\Delta x, \Delta y, \Delta z)$.
Now we can write Eq. (2.13) in a more compact form
that shows $\grad T$ is indeed a vector.
%% Eq. (2.15)
\begin{equation}
  \Delta T = \grad T \cdot \Delta \bv{R}
\end{equation}

But what if we use a different coordinate system?
We need to show that the components of the vector $\grad T$
transform in the same way as the components of the vector $\Delta \bv{R}$.

%% Extra figure, supports Eqs. (2.16), (2.17).
\setlength{\unitlength}{1in} 
\begin{picture}(4,3)(-0.4,0.0)
\thinlines
%\multiput(0.0,0.0)(0.0,0.1){31}{\line(1,0){4.0}}
%\multiput(0.0,0.0)(0.1,0.0){41}{\line(0,1){3.0}}
\put(1.0,0.0){\vector(-1, 3){0.9}} % y' axis
\put(1.0,0.0){\vector( 0, 1){3.0}} % y axis
\put(1.0,0.0){\vector( 3, 1){2.7}} % x' axis
\put(1.0,0.0){\vector( 1, 0){3.0}} % x axis
\put(0.6,1.2){\line( 3, 1){2.4}} % x' extender
\put(1.0,2.0){\line( 1, 0){2.0}} % x extender
\put(3.0,2.0){\line( 1,-3){0.4}} % y' extender
\put(3.05,2.05){$P$}
\put(2.0,2.05){$x$}
\put(1.6,1.6){$x'$}
\put(2.85,1.1){$y$}
\put(3.25,1.4){$y'$}
\thicklines
\put(1.0,0.0){\line( 1, 0){2.0}}
\put(1.0,0.0){\line( 3, 1){1.8}}
\put(3.0,0.0){\line(-1, 3){0.6}}
\put(3.0,0.0){\line( 0, 1){2.0}}
\put(2.4,1.8){\line( 3, 1){0.6}}
\put(1.6,0.05){$\theta$}
\put(2.9,0.4){$\theta$}
\put( 2.60, 1.70){\line(-3,-1){0.150}} % Right-angle symbol in upper corner
\put( 2.60, 1.70){\line(-1, 3){0.050}} % 
\put( 2.70, 0.40){\line( 3, 1){0.150}} % Right-angle symbol in lower corner
\put( 2.70, 0.40){\line(-1, 3){0.050}} %
\end{picture}

\vspace{4ex}
This figure shows the point $P$ in two different coordinate systems.
In one system $P$ has the coordinates $(x,y)$.
In the other system $P$ has the coordinates $(x',y')$.
We use simple trigonometry to derive equations for $x'$ and $y'$
in terms of $x$ and $y$.
Consider the two right triangles, both drawn with thick lines.
One has hypotenuse $x$ and the other has hypotenuse $y$.
The figure shows $x'$ as a sum of two lengths
and $y'$ as a difference of two lengths.
%% Eq. (2.16)
\begin{equation}
  x' = x \cos \theta + y \sin \theta
\end{equation}
%% Eq. (2.17)
\begin{equation}
  y' = y \cos \theta - x \sin \theta
\end{equation}

%% Extra figure, supports Eqs. (2.18), (2.19).
\setlength{\unitlength}{1in} 
\begin{picture}(4,3)(-0.4,0.0)
\thinlines
%\multiput(0.0,0.0)(0.0,0.1){31}{\line(1,0){4.0}}
%\multiput(0.0,0.0)(0.1,0.0){41}{\line(0,1){3.0}}
\put(1.0,0.0){\vector(-1, 3){0.9}} % y' axis
\put(1.0,0.0){\vector( 0, 1){3.0}} % y axis
\put(1.0,0.0){\vector( 3, 1){2.7}} % x' axis
\put(1.0,0.0){\vector( 1, 0){3.0}} % x axis
\put(0.6,1.2){\line( 3, 1){2.4}} % x' extender
\put(1.0,2.0){\line( 1, 0){2.0}} % x extender
\put(3.0,0.0){\line( 0, 1){2.0}} % y extender
\put(3.0,2.0){\line( 1,-3){0.4}} % y' extender
\put(3.05,2.05){$P$}
\put(2.0,2.05){$x$}
\put(1.6,1.6){$x'$}
\put(2.85,0.9){$y$}
\put(3.25,1.4){$y'$}
\thicklines
\put(0.6,1.2){\line( 1, 0){2.4}}
\put(0.6,1.2){\line( 3, 1){2.4}}
\put(3.0,1.2){\line( 0, 1){0.8}}
\put(1.0,0.0){\line( 0, 1){1.2}}
\put(1.0,0.0){\line(-1, 3){0.4}}
\put(0.90,0.40){$\theta$}
\put(1.10,1.25){$\theta$}
\put( 0.90, 1.10){\line( 0, 1){0.100}} % Right-angle symbol on small triangle
\put( 0.90, 1.10){\line( 1, 0){0.100}} % 
\put( 2.90, 1.30){\line( 0,-1){0.100}} % Right-angle symbol on large triangle
\put( 2.90, 1.30){\line( 1, 0){0.100}} %
\end{picture}

\vspace{4ex}
This figure shows the same point $P$ in the two different coordinate systems.
However, we now derive equations for $x$ and $y$ in terms of $x'$ and $y'$.
Consider the two different right triangles.
One has hypotenuse $x'$ and the other has hypotenuse $y'$.
The figure shows $x$ as a diffence of two lengths
and $y$ as a sum of two lengths.
%% Eq. (2.18)
\begin{equation}
  x = x' \cos \theta - y' \sin \theta
\end{equation}
%% Eq. (2.19)
\begin{equation}
  y = y' \cos \theta + x' \sin \theta
\end{equation}

\vspace{4ex}
Figure 2-6(b) shows $P_1$ and $P_2$ in two different coordinate systems
$(x,y)$ and $(x',y')$.  There is no third dimension $z$, so $\Delta z = 0$.
$P_1$ and $P_2$ are chosen such that $\Delta y = 0$ also.
From Eq. (2.13) we now have
%% Eq. (2.20)
\begin{equation}
  \Delta T = \dfdx{T}{x} \Delta x
\end{equation}

We can also write $\Delta T$ in terms of the second coordinate system.
%% Eq. (2.21)
\begin{equation}
  \Delta T = \dfdx{T}{x'} \Delta x' + \dfdx{T}{y'} \Delta y'
\end{equation}

Looking at Figure 2-6(b) we can write values for $\Delta x'$ and $\Delta y'$.
%% Eq. (2.22)
\begin{equation}
  \Delta x' = \Delta x \cos \theta
\end{equation}
%% Eq. (2.23)
\begin{equation}
  \Delta y' =  -\Delta x \sin \theta
\end{equation}

We substitute these two values into Eq. (2.21).
The result looks a lot like Eq. (2.20)
%% Eqs. (2.24) and (2.25)
\begin{eqnarray}
  \Delta T &=& \dfdx{T}{x'} \Delta x \cos \theta - \dfdx{T}{y'} \Delta x \sin \theta \\
  &=& \Bigg( \dfdx{T}{x'} \cos \theta - \dfdx{T}{y'} \sin \theta \Bigg) \Delta x
\end{eqnarray}

except that\ldots
%% Eq. (2.26)
\begin{equation}
  \dfdx{T}{x} = \dfdx{T}{x'} \cos \theta - \dfdx{T}{y'} \sin \theta
\end{equation}

So $\grad T$ is definitely a vector.

