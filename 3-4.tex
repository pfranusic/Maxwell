%% 3-4.tex
\section{Heat conduction; the diffusion equation}

We develop the \emph{heat diffusion equation}.
It is a differential equation in $x, y, z,$ and $t$ for temperature $T$.
We begin with an equation for the ``heat out'' of a cube that is cooling off.
We take equation (3.17) and replace vector $\bv{C}$ with heat flux vector $\bv{h}$.
%% Eq. (3.19)
\begin{equation}
  \mathrm{Heat\ out} = \int\limits_{\mathrm{cube}}
  \bv{h} \cdot \bv{n} \; da = \Big( \grad \cdot \bv{h} \Big) \Delta V
\end{equation}

Next, we write an equation for the ``heat lost'' from the inside of the cube.
There are neither heat sources nor heat sinks inside the cube.
If the heat per unit volume is $q$ and the volume of the cube is $\Delta V$,
then the heat lost is simply the time derivative of the total heat $q\;\Delta V$.
%% Eq. (3.20)
\begin{equation}
  \mathrm{Heat\ lost} = - \frac{d}{dt} (q \; \Delta V) = - \frac{dq}{dt} \; \Delta V
\end{equation}

The conservation of energy law tells us that the heat lost from inside the cube
is exactly the same as the heat out of the cube.
This gives us a \emph{differential} form that appears often in physics.
%% Eq. (3.21)
\begin{eqnarray}
  \mathrm{Heat\ lost} &=& \mathrm{Heat\ out} \phantom{\int} \nonumber\\
  - \frac{dq}{dt} \; \Delta V &=& \grad \cdot \bv{h} \; \Delta V \nonumber\\
  - \frac{dq}{dt} &=& \grad \cdot \bv{h}
\end{eqnarray}

We can also derive the \emph{integral} form of equation (3.13).
We start with Gauss' Theorem and replace vector $\bv{C}$ with heat flux vector $\bv{h}$.
%% Eq. (3.22)
\begin{equation}
  \int_S \bv{h} \cdot \bv{n} \; da = \int_V \grad \cdot \bv{h} \; dV
\end{equation}

We replace the $\grad \cdot \bv{h}$ in the right-hand integral with $-dq/dt$
and then simplify using $\int dV = \Delta V$ and $Q = q\;\Delta V$.
\begin{equation*}
  \int_V \grad \cdot \bv{h} \; dV
  = \int_V \Big(-\frac{dq}{dt}\Big) \; dV
  = - \frac{dQ}{dt}
\end{equation*}

We now consider a different case, where heat energy is generated 
at some point $P$ inside a block of material.
Let $W$ represent the energy liberated per second at that point.
We wish to describe the heat vector field near $P$.
Our ``physical intuition'' tells us that $\bv{h}$ is radial.
That is, each vector $\bv{h}$ points away from $P$.
We imagine a spherical surface $S$ inside the material, 
centered around $P$ and with radius $R$.
This allows us to write a simple equation for the heat flux,
where $4 \pi R^2$ is the area of the sphere.
Note that the dot on the left-hand side is the dot product operator
whereas the dot on the right-hand side is the multiplication operator.
And $h$ is the magnitude of $\bv{h}$.
%% Eq. (3.23)
\begin{equation}
  \int_S \bv{h} \cdot \bv{n} \; da = h \cdot 4 \pi R^2
\end{equation}

The right-hand side is equal to $W$ so we can specify magnitude $h$ in terms of $W$.
Since both the heat vector $\bv{h}$ and the unit vector $\bv{e}_r$ are radial,
we can multiply the unit vector $\bv{e}_r$ by magnitude $h$ to get an equation for vector $\bv{h}$.
%% Eq. (3.24)
\begin{eqnarray}
  h &=& \frac{W}{4 \pi R^2} \nonumber\\
  \bv{h} &=& \frac{W}{4 \pi R^2} \bv{e}_r
\end{eqnarray}

We wish to find an equation for the most general kind of heat flow,
with only the condition that heat is conserved.
We start with equation (2.44), which specifies the differential equation for heat conduction.
Remember that heat flows ``downhill'' (hence the minus sign)
and $\kappa$ is the thermal conductivity constant.
%% Eq. (3.25)
\begin{equation}
  \bv{h} = - \kappa \; \grad T
\end{equation}

Now we take equation (3.21) and multiply both sides by $-1$.
Then we use equation (3.25) to replace $\bv{h}$ with $-\kappa\;\grad T$.
Finally, we rearrange, cancel the two negative signs,
and replace $\grad \cdot \grad$ with the Laplacian operator $\nabla^2$.
%% Eq. (3.26)
\begin{eqnarray}
  \frac{dq}{dt} &=& - \grad \cdot \bv{h} \nonumber\\
  &=& - \grad \cdot (- \kappa \; \grad T) \phantom{\int} \nonumber\\
  &=& \kappa \; \nabla^2 T
\end{eqnarray}

\newpage
We now make the assumption that a small change in the heat energy of the material
is proportional to a small change in the temperature of the material, by some constant $c_v$.
That is, $\Delta q = c_v \; \Delta T$.
We differentiate both sides with respect to time.
%% Eq. (3.27)
\begin{equation}
  \frac{dq}{dt} = c_v \frac{dT}{dt}
\end{equation}

We replace $dq/dt$ with $(\kappa\;\nabla^2 T)$ and rearrange.
We may also replace the constant $\kappa / c_v$ with the diffusion symbol $D$.
%% Eqs. (3.28), (3.29)
\begin{eqnarray}
  \frac{dT}{dt} &=& \frac{\kappa}{c_v} \nabla^2 T \\
  &=& D \; \nabla^2 T
\end{eqnarray}

This is the \emph{heat diffusion equation}.
It is a differential equation in $x, y, z,$ and $t$ for temperature $T$.
We expand it here to show everything.
\begin{equation*}
  \dfdx{T}{t} = \frac{\kappa}{c_v}
  \Bigg( \dfdx{^2T}{x^2} + \dfdx{^2T}{y^2} + \dfdx{^2T}{z^2} \Bigg)
\end{equation*}

