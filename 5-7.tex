%% 5-7.tex
\section{A sphere of charge; a spherical shell}

We can use Gauss' law to find the electric field inside a sphere of uniform charge.
Figure 5-8 shows a sphere with radius $R$ and uniform charge density,
and $\rho$ is the charge per unit volume.

Using arguments of symmetry,
we assume the electric field is radial at every point inside the sphere.
We also assume the electric field has the same magnitude at all points
equidistant from the center of the sphere.

The Gaussian surface is a smaller concentric sphere with radius $r$.
We wish to know the electric field magnitude $E$ outside the smaller sphere.
The flux out of the smaller sphere is simply it's surface area
times the field magnitude, or $4\pi r^2 E$.
The charge inside the smaller sphere is simply it's volume times
the charge per unit volume, or $\frac{4}{3} \pi r^3 \rho$.
Gauss's law says that the flux out of the smaller sphere is equal to
the charge inside, divided by $\epsilon_0$.
%% nonumber and Eq. (5.7)
\begin{equation*}
  4\pi r^2 E  = \frac{\frac{4}{3} \pi r^3 \rho}{\epsilon_0}
\end{equation*}

Dividing both sides by $4 \pi r^2$ we get
\begin{equation}
  E = \frac{\rho r}{3 \epsilon_0} \qquad (r < R)
\end{equation}

Now we consider a thin spherical shell of charge.
Using arguments of symmetry,
we assume the electric field is radial at every point inside and outside the shell.
We also assume the electric field has the same magnitude at all points 
equidistant from the center of the shell.
The field outside the shell is like a point charge.
The field inside the shell is zero.

