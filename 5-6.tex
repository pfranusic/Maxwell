%% 5-6.tex
\section{A sheet of charge; two sheets}

Now we'll calculate the field produced by a sheet of charge.
The sheet is an infinite plane and the charge is distributed uniformly.
Let $\sigma$ represent the charge per unit area.

Using arguments of symmetry,
we assume the electric field is normal to the plane at every point near the plane.
We also assume the electric field has the same magnitude on each side of the plane.

Figure 5-6 shows part of a \textsc{uniformly charged sheet} and a \textsc{gaussian surface}.
This is a box that extends on both sides of the sheet.
Four sides of the box are perpendicular to the sheet.
The other two sides are parallel to the sheet.
The two sides each have area $A$.
One side has electric field $\bv{E}_1$ and the opposite side $\bv{E}_2$.

We use Gauss' law to equate the flux to the charge inside the Gaussian surface.
We make several assumptions:
that there are no other charges except those in the plane,
that both $\bv{E}_1$ and $\bv{E}_2$ are normal to the plane,
and that the $\bv{E}_1$ and $\bv{E}_2$ have equal magnitude $E$.
Therefore the flux through each parallel face is $EA$.
The charge on the part of the sheet enclosed by the box is $\sigma A$.
%% nonumber
\begin{equation*}
  EA + EA = \frac{\sigma A}{\epsilon_0}
\end{equation*}

Dividing both sides by $2A$ we have
%% Eq. (5.3)
\begin{equation}
  E = \frac{\sigma}{2 \epsilon_0}
\end{equation}

If there are other charges besides those on the sheet,
then $\bv{E}_1$ and $\bv{E}_2$ will have different magnitudes.
In the simplification, we can only divide both sides by $A$.
%% Eq. (5.4)
\begin{equation}
  E_1 + E_2 = \frac{\sigma}{\epsilon_0}
\end{equation}

Next we consider two sheets of charge.
The two sheets are parallel planes and contain opposite charges.
Let $+\sigma$ and $-\sigma$ be the two charge densities.
Using arguments of symmetry, we assume the electric field is normal to each plane.

Figure 5-7(b) shows a side view of the two sheets with a Gaussian surface 
that surrounds part of the negative sheet, similar to Figure 5-6.
According to Equation (5.3), this produces the electric field 
magnitude $-E = -\sigma / 2 \epsilon_0$ (or simply $E = \sigma / 2 \epsilon_0$).
Figure 5-7(b) shows the same thing for the positive sheet.
Here, the electric field magnitude is $+E = +\sigma / 2 \epsilon_0$.
We simply add the two fields to get $E$ between the sheets.
%% Eqs. (5.5) and (5.6)
\begin{equation}
  E_{\mathrm{between}} = \sigma / \epsilon_0
\end{equation}

Figure 5-7(a) shows a side view of the two sheets and a Gaussian surface.
Again, this is a box, except that it surrounds equal parts of each sheet.
Four sides are perpendicular to the sheets and the remaining two sides 
are parallel to the sheets.
It is easy to see that the box contains equal and opposite charges that cancel each other.
Therefore the electric field outside the box is zero.
\begin{equation}
  E_{\mathrm{outside}} = 0
\end{equation}

