%% 5-2.tex
\section{Equilibrium in an electrostatic field}

This section explores the idea of equilibrium in several different situations
and demonstrates that there are \emph{no} positions of stable equilibrium
within \emph{any} electrostatic field. (We disregard the positions of point charges.)

The first situation is a plane in free space.
The plane contains three negative charges which are equidistant from each other.
I.e., the three negative charges are at the corners of an equilateral triangle.
Now we place a positive charge at the exact center of the triangle.
It does not move because the force is zero
\emph{at this singular point in the middle of the triangle.}
But this isn't a position of stable equilibrium.
If we nudge the positive charge ever so slightly, it won't return to the center.

In order for a charge to be in stable equilibrium at some position $P_0$ 
within an electrostatic field, there are two conditions that would have to be met.
The first condition is that the field at position $P_0$ must be zero.
The second condition is that \emph{all} of the electric field vectors at nearby positions
must be aimed inward towards $P_0$. But this condition violates Gauss' law,
and the next situation demonstrates why.

Figure 5-1 shows a cutaway of an imaginary spherical surface that surrounds $P_0$.
Per the second condition of equilibrium, all of the nearby vectors are aimed at $P_0$.
This means that the surface integral would be some negative number.
But if there is no charge at $P_0$, the surface integral would have to be zero, by Gauss' law.
We have a contradiction here, and since Gauss' law is inviolable, an imaginary surface
with vectors aimed inward must be an impossibility. The second condition is impossible.

\textcolor{red}{The next situation is two equal charges fixed on a rod.
But I'm having trouble visualizing the force $\bv{F}$ on the rod in any position.
And which argument is referenced by ``following the argument used above\ldots''?}
The Lorentz force is given in Eq. (1.1):
\begin{equation*}
  \bv{F} = q(\bv{E} + \bv{v} \times \bv{B})
\end{equation*}

The Lorentz force for static charges is simply $\bv{F} = q\bv{E}$
because the velocity vector $\bv{v}$ is $(0,0,0)$ and the cross-product
cancels any effects from the magnetic field vector $\bv{B}$.
The superposition principle says that we can add the Lorentz force
of each charge to get the total force on the rod.
%% Eq. (5.1) and nonumber
\begin{equation}
  \bv{F} = q_1 \bv{E}_1 + q_2 \bv{E}_2
\end{equation}

Now we differentiate both sides and rearrange the right side.
This gives the separate factors $(\grad \cdot \bv{E}_1)$ and $(\grad \cdot \bv{E}_2)$.
\begin{equation*}
  \grad \cdot \bv{F} =
  q_1 \left( \grad \cdot \bv{E}_1 \right) +
  q_2 \left( \grad \cdot \bv{E}_2 \right) \nonumber
\end{equation*}

\textcolor{red}{I don't understand why ``both $\grad \cdot \bv{E}_1$ and 
$\grad \cdot \bv{E}_2$ are zero.''}
It follows that the divergence $\grad \cdot \bv{F}$ is therefore zero.
But the argument of Figure 5-1 demonstrates that the divergence must be negative
for stable equilibrium to exist.
