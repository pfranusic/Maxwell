%% 3-8.tex
\section{Summary}

\textsc{The $\grad$ Operator}:\\
The operators $\partial/\partial x$, $\partial/\partial y$, and $\partial/\partial z$
can be considered as the three components of a vector operator $\grad$,
and the formulas which result from vector algebra by treating this operator
as a vector are correct.
%% nonumber
\begin{equation*}
  \grad = \Bigg( \dfdx{}{x} , \dfdx{}{y} , \dfdx{}{z} \Bigg)
\end{equation*}

\textsc{The Fundamental Theorem of Calculus}:\\
The difference of the values of a scalar field at two points is equal to the
line integral of the tangential component of the gradient of that scalar along
any curve at all between the first and second points.
%% Eq. (3.42)
\begin{equation}
  \psi(2) - \psi(1) = \int
  _{\hspace{-3ex}\raisebox{-1ex}{\mbox{$\genfrac{}{}{0pt}{1}{(1)}{\mathrm{any\ curve}}$}}}
  ^{(2)} \grad \psi \cdot d\bv{s} 
\end{equation}

\textsc{Gauss' Theorem}:\\
The surface integral of the normal component of an arbitrary vector
over a closed surface is equal to the integral of the divergence of the vector over
the volume interior to the surface.
%% Eq. (3.43)
\begin{equation}
  \int\limits_{\genfrac{}{}{0pt}{1}{\mathrm{closed}}{\mathrm{surface}}}
  \bv{C} \cdot \bv{n} \; da =
  \int\limits_{\genfrac{}{}{0pt}{1}{\mathrm{volume}}{\mathrm{inside}}}
  \grad \cdot \bv{C} \; dV
\end{equation}

\textsc{Stokes' Theorem}:\\
The line integral of the tangential component of an arbitrary vector
around a closed loop is equal to the surface integral of the normal component
of the curl of that vector over any surface which is bounded by the loop.
%% Eq. (3.44)
\begin{equation}
  \oint\limits_{\mathrm{boundry}}
  \bv{C} \cdot d\bv{s} = 
  \int\limits_{\mathrm{surface}}
  (\grad \times \bv{C}) \cdot \bv{n} \; da
\end{equation}

