%% 2-2.tex
\section{Scalar and vector fields --- $T$ and $\bv{h}$}

%% intro
Our ultimate goal is to explain the meaning of the laws given in Chapter 1.
But to do this we must understand the mathematics of vector fields.
We first review vector algebra: scalars, vectors, scalar products, and vector products.

%% def of scalar
A \emph{scalar} $m$ is a value that consists of a single component.
The component is an element of the set of real numbers.
\begin{equation*}
  m \in \mathcal{R}
\end{equation*}

%% def of vector
A \emph{vector} $\bv{A}$ is a value that has three components --- $A_x$, $A_y$, and $A_z$ ---
from the rectangular coordinate system.
Each component is an element of the set of real numbers.
\begin{equation*}
  \bv{A} = \Big( \; A_x \; , \; A_y \; , \; A_z \; \Big)
\end{equation*}

%% def of magnitude
The \emph{magnitude} of vector $\bv{A}$ is the scalar $|\bv{A}|$,
sometimes written simply as $A$.
We use the Pythagorean theorem, extended for three dimensions, to calculate the magnitude $|\bv{A}|$.
\begin{equation*}
  |\bv{A}| = \sqrt{ A_x^2 + A_y^2 + A_z^2 }
\end{equation*}

%% def of vector addition
To add two vectors $\bv{A}$ and $\bv{B}$, we simply add their components.
\begin{equation*}
  \bv{A} + \bv{B} = \Big(\big(A_x + B_x\big), \big(A_y + B_y\big), \big(A_z + B_z\big) \Big)
\end{equation*}

%% def of simple product
The product of a scalar $m$ and a vector $\bv{A}$ is a vector.
Each of the three components of $\bv{A}$ is multiplied by $m$.
\begin{equation*}
  m \bv{A} = \Big(\big(m A_x\big),\big(m A_y\big),\big( m A_z\big)\Big)
\end{equation*}

%% Eq. (2.1)
%% def of scalar product
The \emph{scalar product} of two vectors $\bv{A}$ and $\bv{B}$ is a scalar.
The scalar product is sometimes called the \emph{dot product}.
It has two forms.
In the first form, each of the three components of vector $\bv{A}$ are multiplied by 
the three corresponding components of vector $\bv{B}$.
In the second form, the magnitudes of $\bv{A}$ and $\bv{B}$ 
are multiplied by the cosine of the angle $\theta$ that separates $\bv{A}$ and $\bv{B}$.
\begin{eqnarray}
  \bv{A} \cdot \bv{B} &=& \mathrm{scalar} \\
  &=& A_x B_x + A_y B_y + A_z B_z \nonumber \\
  &=& A B \cos \theta \nonumber
\end{eqnarray}

%% Eq. (2.2)
%% def of vector product
The \emph{vector product} of two vectors $\bv{A}$ and $\bv{B}$ is a vector.
The vector product is sometimes called the \emph{cross product}.
Each component of the result is a difference of products.
% The resulting vector $(\bv{A} \times \bv{B})$ is at right angles 
% to both $\bv{A}$ and $\bv{B}$ and has a magnitude of $AB$.
\begin{eqnarray}
  \bv{A} \times \bv{B} &=& \mathrm{vector} \\
  &=& \Big( \big( A_y B_z - A_z B_y \big),
            \big( A_z B_x - A_x B_z \big), 
	    \big( A_x B_y - A_y B_x \big) \Big) \nonumber
\end{eqnarray}

%% Equation (2.3)
The vector product of $\bv{A}$ with itself is simply 
the zero vector $(0,0,0)$ which is represented by the symbol $\boldsymbol{0}$.
The proof is quick.
\begin{equation}
  \bv{A} \times \bv{A} = \boldsymbol{0}
\end{equation}

%% Equations (2.4), (2.5), (2.6).
Here are some identities.
The proofs are simple but somewhat tedious.
%% Eq. (2.4)
\begin{equation}
  \bv{A} \cdot (\bv{A} \times \bv{B}) = 0
\end{equation}
%% Eq. (2.5)
\begin{equation}
  \bv{A} \cdot (\bv{B} \times \bv{C}) = (\bv{A} \times \bv{B}) \cdot \bv{C}
\end{equation}
%% Eq. (2.6)
\begin{equation}
  \bv{A} \times (\bv{B} \times \bv{C}) = \bv{B}(\bv{A} \cdot \bv{C}) - \bv{C}(\bv{A} \cdot \bv{B})
\end{equation}


%% Eq. (2.7)
The \emph{total differential} for three dimensions looks like 
the scalar product of two vectors.
\begin{equation}
  \Delta f(x,y,z) = \dfdx{f}{x} \Delta x + \dfdx{f}{y} \Delta y + \dfdx{f}{z} \Delta z
\end{equation}


%% Eq. (2.8)
When we have a second derivative with two coefficients in the denominator,
the coefficients are commutative.
\begin{equation}
  \frac{\partial ^2 f}{\partial x \partial y} = \frac{\partial ^2 f}{\partial y \partial x}
\end{equation}

%% def of scalar field
A \emph{scalar field} is a set of scalars, where each scalar is associated
with a unique position $(x,y,z)$ in space.
For example, consider a concrete slab, one meter square and ten centimeters thick,
where the top of the slab is heated by the sun and the bottom contacts some frozen ground.
Heat flows through the slab from top to bottom.
At any instant in time, each point inside the slab has a specific temperature $T$
(degrees Celsius), and we have the scalar function $T(x,y,z)$.
This example demonstrates that temperature is a scalar field.

%% def of vector field
A \emph{vector field} is a set of vectors, where each vector is associated
with a unique position $(x,y,z)$ in space.
For example, consider the concrete slab again.
At any instant in time, each point inside the slab has a specific heat flow $\bv{h}$
with a specific rate (Joules per square meter) and a specific direction.
We therefore have the vector function $\bv{h}(x,y,z)$ which specifies
a heat flow vector for each point inside the slab at some instant.
This example demonstrates that heat flow is a vector field.

%% Eq. (2.9)
%% heat flow vector equation
A \emph{unit vector} is a vector with a magnitude of 1.
We can use a unit vector to specify the heat flow vector $\bv{h}$.
Let $\Delta J$ be the amount of thermal energy that passes through a small area $\Delta a$.
Let $\bv{e}_f$ be a unit vector ($f \equiv$ flow) that points in the same direction as $\bv{h}$.
Then the magnitude of $\bv{h}$ is $\Delta J / \Delta a$ and 
\begin{equation}
  \bv{h} = \frac{\Delta J}{\Delta a} \bv{e}_f
\end{equation}

%% Eq. (2.10)
%% Scalar product of heat flow vector and normal unit vector.

We wish to specify how much heat flows through a small surface
at \emph{any} angle with respect to the flow.  For example,
we wish to specify how much heat flows through the surface $\Delta a_2$ in Figure 2-4.

The unit vector $\bv{n}$ is normal to surface $\Delta a_2$.
The heat flow vector $\bv{h}$ is normal to surface $\Delta a_1$
and points in a different direction than unit vector $\bv{n}$.
The two vectors are separated by the angle $\theta$.
The same amount of heat flows through $\Delta a_1$ and $\Delta a_2$.
These two surface areas are related such that $\Delta a_1 = \Delta a_2 \cos \theta$.
For proof, consider the prism implied in Figure 2-4.

%% Extra figure, supports Fig. 2-4.
%% This is a small right triangle.
%% Frame is 5.00 cm wide and 3.00 cm tall.
%% The bottom left-hand corner is at (-3.50, 0.00).
\setlength{\unitlength}{1cm}
\begin{picture}(5.00,3.00)(-3.50, 0.00)
%% Draw the frame boundries.
%% (These may be commented out.)
% \put( 0.00, 3.00){\line( 1, 0){4.75}} % North boundry
% \put( 0.00, 0.00){\line( 1, 0){4.75}} % South boundry
% \put( 4.75, 0.00){\line( 0, 1){3.00}} % West boundry
% \put( 0.00, 0.00){\line( 0, 1){3.00}} % West boundry
%% Draw the triangle.
\put( 1.00, 1.40){\line( 3,-1){2.56}} % Hypotenuse
\put( 1.00, 1.40){\line( 5, 2){2.00}} % Opposite
\put( 3.00, 2.20){\line( 1,-3){0.55}} % Adjacent
%% Draw the various symbols.
\put( 2.73, 1.67){\line( 5, 2){0.400}} % Right-angle symbol
\put( 2.73, 1.67){\line(-1, 3){0.125}} % Right-angle symbol
\put( 3.15, 0.78){$\theta$} % Angle symbol
\put( 1.80, 0.60){$\frac{\Delta a_2}{w}$} % Hypotenuse symbol
\put( 3.35, 1.40){$\frac{\Delta a_1}{w}$} % Adjacent symbol
\end{picture}

The prism has width $w$.
One face of the prism is the surface $\Delta a_1$.
Another face is the surface $\Delta a_2$.
A third face is a right triangle with angle $\theta$,
a hypotenuse $\Delta a_2/w$, and adjacent side $\Delta a_1/w$.
The cosine of $\theta$ is defined as the length of the adjacent side
divided by the length of the hypotenuse.
\begin{eqnarray*}
  \cos \theta &=& \frac{\Delta a_1/w}{\Delta a_2/w} \quad = \quad \frac{\Delta a_1}{\Delta a_2} \\
  \Delta a_1 &=& \Delta a_2 \cos \theta
\end{eqnarray*}

Rearrange to get the surface areas in the denominators.
\begin{equation*}
  \frac{1}{\Delta a_2} \quad = \quad \frac{1}{\Delta a_1} \cos \theta
\end{equation*}

Again we ask, what is the heat flow through $\Delta a_2$?
It is $\Delta J / \Delta a_2$.
And the magnitude of $\bv{h}$ is $\Delta J / \Delta a_1$.
Now we multiply both sides of the equation above by $\Delta J$.
This is the same as the scalar product of $\bv{h}$ and $\bv{n}$.
\begin{equation}
  \frac{\Delta J}{\Delta a_2} \quad = \quad
  \frac{\Delta J}{\Delta a_1} \cos \theta \quad = \quad
  \bv{h} \cdot \bv{n}
\end{equation}

The heat flow through $\Delta a_2$ is the scalar product
of the heat flow vector $\bv{h}$ and the normal unit vector $\bv{n}$.



\begin{comment}
%%%%%%%%%%%%%%%%%%%%%%%%%%%%%%%%%%%%%%%%%%%%%%%%%%%%%%%%%%%%%%%%%%%%%%%%%%%%%%%%

%% Eq. (2.10)
Figure 2-4 in the text shows two small surface areas $\Delta a_1$ and $\Delta a_2$.
Vector $\bv{h}$ is normal to $\Delta a_1$ and vector $\bv{n}$ is normal to $\Delta a_2$.
These two vectors are separated by an angle of $\theta$.
Figure 2-4 shows the surfaces areas as rectangles with the same width.
Let this width be $w$.
Therefore, we can compute the area of each rectangle.
\begin{eqnarray*}
  \Delta a_1 &=& w \cdot \frac{\Delta a_1}{w} \\
  \Delta a_2 &=& w \cdot \frac{\Delta a_2}{w} \\
\end{eqnarray*}

%% Eq. (2.10), continued.
We can construct a right triangle with the lengths of the rectangles.
The figure below shows a right triangle with angle $\theta$,
adjacent side $\frac{\Delta a_1}{w}$, and hypotenuse $\frac{\Delta a_2}{w}$.
The adjacent side represents the length of the rectangle for area $\Delta a_1$.
The hypotenuse represents the length of the rectangle for area $\Delta a_2$.

%% Extra figure, supports Fig. 2-4.
%% Frame is 5.00 cm wide and 3.00 cm tall.
%% The bottom left-hand corner is at (-3.50, 0.00).
\setlength{\unitlength}{1cm}
\begin{picture}(5.00,3.00)(-3.50, 0.00)
%% Draw the frame boundries.
%% (These may be commented out.)
% \put( 0.00, 3.00){\line( 1, 0){4.75}} % North boundry
% \put( 0.00, 0.00){\line( 1, 0){4.75}} % South boundry
% \put( 4.75, 0.00){\line( 0, 1){3.00}} % West boundry
% \put( 0.00, 0.00){\line( 0, 1){3.00}} % West boundry
%% Draw the triangle.
\put( 1.00, 1.40){\line( 3,-1){2.56}} % Hypotenuse
\put( 1.00, 1.40){\line( 5, 2){2.00}} % Opposite
\put( 3.00, 2.20){\line( 1,-3){0.55}} % Adjacent
%% Draw the various symbols.
\put( 2.73, 1.67){\line( 5, 2){0.400}} % Right-angle symbol
\put( 2.73, 1.67){\line(-1, 3){0.125}} % Right-angle symbol
\put( 3.15, 0.78){$\theta$} % Angle symbol
\put( 1.80, 0.60){$\frac{\Delta a_2}{w}$} % Hypotenuse symbol
\put( 3.35, 1.40){$\frac{\Delta a_1}{w}$} % Adjacent symbol
\end{picture}

%% Eq. (2.10), continued.
The cosine of angle $\theta$ is the length of the adjacent side $\Delta a_1 / w$
divided by the length of the hypotenuse $\Delta a_2 / w$.
We cancel the two $w$ terms and multiply both sides by $\Delta a_2$.
\begin{equation*}
  \Delta a_1 = \Delta a_2 \cos \theta
\end{equation*}

%% Eq. (2.10), continued.
The same amount of thermal energy $\Delta J$ passes through surface 
$\Delta a_2$ as passes through $\Delta a_1$.
We divide $\Delta J$ by $\Delta a_1$ on one side and 
by $\Delta a_2 \cos \theta$ on the other, then rearrange.
\begin{equation*}
  \frac{\Delta J}{\Delta a_2} = \frac{\Delta J}{\Delta a_1} \cos \theta
\end{equation*}

%% Eq. (2.10), continued.
The magnitude of vector $\bv{h}$ is, by definition, $h = \Delta J / \Delta a_1$.
The unit vector $\bv{n}$ is normal to the surface $\Delta a_2$
and its magnitude is $n = 1$.
The scalar product of the two vectors is
\begin{equation*}
  \bv{h} \cdot \bv{n} = h n \cos \theta = \frac{\Delta J}{\Delta a_1} \cos \theta
\end{equation*}

%% full presentation of Eq. (2.10)
To summarize, the heat flow --- per unit time and per unit area through surface area $\Delta a_2$ 
with unit normal $\bv{n}$ --- is given by the scalar product $\bv{h} \cdot \bv{n}$ ---
where $\bv{h}$ represents the magnitude and directon of the heat flow that is
normal to surface area $\Delta a_1$.
\begin{equation}
  \frac{\Delta J}{\Delta a_2} = \frac{\Delta J}{\Delta a_1} \cos \theta = \bv{h} \cdot \bv{n}
\end{equation}

%%%%%%%%%%%%%%%%%%%%%%%%%%%%%%%%%%%%%%%%%%%%%%%%%%%%%%%%%%%%%%%%%%%%%%%%%%%%%%%%
\end{comment}

