%% 4-4.tex
\section{$\bv{E} = - \grad \phi$}

This section makes three points.
The first is that $\bv{E}$ can be easily obtained from $\phi$.
We merely need to calculate the derivative.
That is, $\bv{E}$ is the vector derivative of $\phi$.
We first express $\Delta W$ as the difference in potential energy
between two points $(x,y,z)$ and $(x+\Delta x,y,z)$.
\textcolor{red}{How is this justified?}
We then use a partial derivative to approximate $\Delta W$ for small $\Delta x$.
\textcolor{red}{And this?}
%% nonumber
\begin{eqnarray*}
  \Delta W &=& \phi(x + \Delta x, y, z) - \phi(x,y,z) \\
  &=& \dfdx{\phi}{x} \Delta x
\end{eqnarray*}

We can also use Eq. (4.19) to express $\Delta W$ as an integral.
\textcolor{red}{Then this is \emph{somehow} simplified with $- E_x \Delta x$.}
%% nonumber
\begin{eqnarray*}
  \Delta W &=& - \int \bv{E} \cdot d\bv{s} \\
  &=& -E_x \Delta x
\end{eqnarray*}

These first equation for $\Delta W$ has $\partial \phi/\partial x$ as the coefficient for $\Delta x$.
The second equation has $-E_x$. So they must be equal.
We use the same logic for $E_y$ and $E_z$.
%% Eq. (4.26)
\begin{eqnarray}
  E_x &=& - \dfdx{\phi}{x} \\
  E_y &=& - \dfdx{\phi}{y} \nonumber\\
  E_z &=& - \dfdx{\phi}{z} \nonumber
\end{eqnarray}

We now use the $\grad$ operator to collapse these three equations into one.
%% Eq. (4.27)
\begin{equation}
  \bv{E} = - \grad \phi
\end{equation}

Note that the work $\Delta W$ to move a unit charge from $a$ to $b$
can also be expressed in the same form as the Fundamental Theorem.
%% Eq. (4.28)
\begin{equation}
  \int_a^b \grad \phi \cdot d\bv{s} = \phi(b) - \phi(a)
\end{equation}

The second point is that in electrostatics, the electric field is curl-free.
In the cross-product $\grad \times \bv{E}$ we replace the $\bv{E}$ with $-\grad\phi$
which gives us $-(\grad\times\grad)\phi$.
Since the cross-product of any vector with itself is the zero vector $\bv{0}=(0,0,0)$
we now have
%% Eq. (4.29)
\begin{equation}
  \grad \times \bv{E} = \bv{0}
\end{equation}
%\begin{eqnarray}
%  \grad \times \bv{E}
%  &=& \grad \times (-\grad\phi) \nonumber\\
%  &=& - (\grad \times \grad)\phi \nonumber\\
%  &=& (0,0,0) \nonumber\\
%  \grad \times \bv{E} &=& 0
%\end{eqnarray}

The third point is that for any \emph{radial} force
the work done is independent of the path, and there exists a potential.
For example, the electrical force from a single charge is 
radial and spherically symmetrical.

