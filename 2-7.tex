%% 2-7.tex
\section{Second derivatives of vector fields}

In this section we explore the various possible second derivatives.
We are given three expressions:
(1) the gradient $\grad T$, 
(2) the dot product $\grad \cdot \bv{h}$, and 
(3) the cross product $\grad \times \bv{h}$.
We then take each one of these and try to compute the gradient,
the dot product, and the vector product with $\grad$.
Here are all nine possible combinations.
Two result in a scalar, three in a vector, and four don't make any sense.
\begin{eqnarray*}
  && \grad (\grad T) \to \mathrm{??} \\
  && \grad \cdot (\grad T) \to \mathrm{scalar} \\
  && \grad \times (\grad T) \to \mathrm{vector} \\
  && \grad (\grad \cdot \bv{h}) \to \mathrm{vector} \\
  && \grad \cdot (\grad \cdot \bv{h}) \to \mathrm{??} \\
  && \grad \times (\grad \cdot \bv{h}) \to \mathrm{??} \\
  && \grad (\grad \times \bv{h}) \to \mathrm{??} \\
  && \grad \cdot (\grad \times \bv{h}) \to \mathrm{scalar} \\
  && \grad \times (\grad \times \bv{h}) \to \mathrm{vector} \\
\end{eqnarray*}

We dispense with the four expressions that don't make sense.
The expression $\grad (\grad T)$ says to compute the gradient of gradient $\grad T$.
This doesn't make sense because $\grad T$ is a vector 
and we can only compute the gradient of a scalar.
The expression $\grad (\grad \times \bv{h})$ doesn't make sense for the same reason, 
because $(\grad \times \bv{h})$ is a vector.
The expression $\grad \cdot (\grad \cdot \bv{h})$ says to 
compute the dot product of vector $\grad$ and scalar $(\grad \cdot \bv{h})$.
This doesn't make sense because we can only compute the dot product of two vectors.
The expression $\grad \times (\grad \cdot \bv{h})$ says to
compute the cross product of vector $\grad$ and scalar $(\grad \cdot \bv{h})$.
This doesn't make sense because we can only compute the cross product of two vectors.

Out of the nine possible combinations, only five make sense.
These are labeled (a) through (e) and summarized in the equation below.
The implications of each expression is discussed in the remainder of this section.
%% Eq. (2.45)
\begin{eqnarray}
  \mathrm{(a)} & & \grad \cdot \big( \grad T \big) \nonumber \\
  \mathrm{(b)} & & \grad \times \big( \grad T \big) \nonumber \\
  \mathrm{(c)} & & \grad \big( \grad \cdot \bv{h} \big) \\
  \mathrm{(d)} & & \grad \cdot \big( \grad \times \bv{h} \big) \nonumber \\
  \mathrm{(e)} & & \grad \times \big( \grad \times \bv{h} \big) \nonumber
\end{eqnarray}

Expression (b), $\grad \times \big( \grad T \big)$, is the curl of the gradient of $T$.
The curl of any gradient is the zero vector $\bv{0} = (0,0,0)$.
%% Eq. (2.46)
\begin{equation}
  \mathrm{curl\ } \big( \mathrm{grad\ } T \big) = \grad \times \big( \grad T \big) = \bv{0}
\end{equation}

The reason is easily demonstrated.
Consider the $z$ component of the cross product.
The two partial derivatives will always be identical to each other
because of the identity in equation (2.8), so the subtraction will always result in 0.
%% Eq. (2.47)
\begin{eqnarray}
  \big[ \grad \times (\grad T) \big]_z
  &=& \nabla_x (\grad T)_y - \nabla_y (\grad T)_x \nonumber \\
  &=& \dfdx{}{x} \Bigg( \dfdx{T}{y} \Bigg) - \dfdx{}{y} \Bigg( \dfdx{T}{x} \Bigg)
\end{eqnarray}

Expression (d), $\grad \cdot (\grad \times \bv{h})$, has the same form as equation (2.4).
The proof is quick.
%% Eq. (2.48)
\begin{equation}
  \bv{A} \cdot (\bv{A} \times \bv{B}) = 0
\end{equation}

We simply replace $\bv{A}$ with $\grad$ and $\bv{B}$ with $\bv{h}$.
The divergence of the curl of any vector $\bv{h}$ is zero.
%% Eq. (2.49)
\begin{equation}
  \grad \cdot (\grad \times \bv{h}) = \mathrm{div\ } (\mathrm{curl\ } \bv{h}) = 0
\end{equation}

\newpage
Expressions (b) and (d) allow us to make two theorems.
The first theorem is derived from equation (2.46)
except that it uses the scalar $\psi$ instead of the scalar $T$.
The theorem turns things around a little.
It says that if the curl of some vector $A$ is zero,
then $A$ must be the gradient of some scalar $\psi$.
%% Eq. (2.50)
\begin{eqnarray}
  & &  \mathbf{Theorem:} \phantom{XXXXXXXXXXXXXXXXXXXXXXXX} \nonumber\\
  & &  \mathrm{If\ } \quad \grad \times \bv{A} = 0 \nonumber\\
  & &  \mathrm{there\ is\ a\ } \quad \psi \nonumber\\
  & &  \mathrm{such\ that\ } \quad \bv{A} = \grad \psi
\end{eqnarray}

The second theorem is derived from equation (2.49)
except that it uses the vector $\bv{C}$ instead of the vector $\bv{h}$.
Again, things are turned around a little.
The theorem says that if the divergence of some vector $D$ is zero,
then $D$ must be the curl of some vector $\bv{C}$.
%% Eq. (2.51)
\begin{eqnarray}
  & &  \mathbf{Theorem:} \phantom{XXXXXXXXXXXXXXXXXXXXXXXX} \nonumber\\
  & &  \mathrm{If\ } \quad \grad \cdot \bv{D} = 0 \nonumber\\
  & &  \mathrm{there\ is\ a\ } \quad \bv{C} \nonumber\\
  & &  \mathrm{such\ that\ } \quad \bv{D} = \grad \times \bv{C}
\end{eqnarray}

Expression (a), $\grad \cdot (\grad T)$, doesn't always result in 0, so we expand it.
The three components of $\grad$ are $\nabla_x$, $\nabla_y$, and $\nabla_z$.
The three components of $\grad T$ are $(\nabla_x T)$, $(\nabla_y T)$, and $(\nabla_z T)$.
%% Eq. (2.52)
\begin{eqnarray}
  \grad \cdot (\grad T)
  &=& \nabla_x ( \nabla_x T) + \nabla_y ( \nabla_y T) + \nabla_z ( \nabla_z T) \nonumber \\
  &=& \dfdx{^2T}{x^2} + \dfdx{^2T}{y^2} + \dfdx{^2T}{z^2}
\end{eqnarray}

Because multiplication is associative, we can move the parentheses around.
We then introduce a new symbol, $\nabla^2$.
%% Eq. (2.53)
\begin{equation}
  \grad \cdot (\grad T) = \grad \cdot \grad T = (\grad \cdot \grad) T = \nabla^2 T
\end{equation}

The symbol $\nabla^2$ is called the \emph{Laplacian} and it's a scalar operator.
Just like $\grad$, it's hungry for something to differentiate.
%% Eq. (2.54)
\begin{equation}
  \mathrm{Laplacian\ } = \nabla^2 = \dfdx{^2}{x^2} + \dfdx{^2}{y^2} + \dfdx{^2}{z^2}
\end{equation}

Expression (e), $\grad \times (\grad \times \bv{h})$, has the same form as equation (2.6).
%% Eq. (2.55)
\begin{equation}
  \bv{A} \times (\bv{B} \times \bv{C})
  = \bv{B} (\bv{A} \cdot \bv{C}) - \bv{C}(\bv{A} \cdot \bv{B})
\end{equation}

The second term on the right side, $\bv{C}(\bv{A} \cdot \bv{B})$,
is the arithmetic product of vector $\bv{C}$ and scalar $\bv{A} \cdot \bv{B}$.
Since multiplication is commutative, we can replace this term
with $(\bv{A} \cdot \bv{B}) \bv{C}$.
%% Eq. (2.56)
\begin{equation}
  \bv{A} \times (\bv{B} \times \bv{C})
  = \bv{B} (\bv{A} \cdot \bv{C}) - (\bv{A} \cdot \bv{B}) \bv{C}
\end{equation}

Now we replace both $\bv{A}$ and $\bv{B}$ with $\grad$.
We also replace $\bv{C}$ with $\bv{h}$.
%% Eq. (2.57)
\begin{equation}
  \grad \times (\grad \times \bv{h})
  = \grad (\grad \cdot \bv{h}) - (\grad \cdot \grad) \bv{h}
\end{equation}

We can use the Laplacian in the second term on the right side.
%% Eq. (2.58)
\begin{equation}
  \grad \times (\grad \times \bv{h})
  = \grad (\grad \cdot \bv{h}) - \nabla^2 \bv{h}
\end{equation}

Expression (c), $\grad (\grad \cdot h)$, is just some vector.
There isn't anything remarkable about it.
We now summarize things.
We also add expression (f), the Laplacian of vector $\bv{h}$.
The vector $\nabla^2 \bv{h}$ has three components: $(\nabla^2 \bv{h}_x)$,
$(\nabla^2 \bv{h}_y)$, and $(\nabla^2 \bv{h}_z)$.
%% Eq. (2.59)
\begin{eqnarray}
  \mathrm{(a)} & & \grad \cdot (\grad T) = \nabla^2 T = \mathrm{a\ scalar\ field} \nonumber\\
  \mathrm{(b)} & & \grad \times (\grad T) = (0,0,0) \nonumber\\
  \mathrm{(c)} & & \grad (\grad \cdot \bv{h}) = \mathrm{a\ vector\ field} \nonumber\\
  \mathrm{(d)} & & \grad \cdot (\grad \times \bv{h}) = 0 \\
  \mathrm{(e)} & & \grad \times (\grad \times \bv{h}) = 
               \grad (\grad \cdot \bv{h}) - \nabla^2 \bv{h} \nonumber\\
  \mathrm{(f)} & & (\grad \cdot \grad)\bv{h} =
               \nabla^2 \bv{h} = \mathrm{a\ vector\ field} \nonumber
\end{eqnarray}

The dot product $(\grad \cdot \grad)$ makes sense and defines the Laplacian.
But the cross product $(\grad \times \grad)$ doesn't make sense,
because of the identity in equation (2.3), $\bv{A} \times \bv{A} = 0$.

