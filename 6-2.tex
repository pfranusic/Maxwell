%% 6-2.tex
\section{The electric dipole}

An electric dipole consists of two equal and opposite point charges $+q$ and $-q$
separated by a very small distance $d$. (See Figure 6-1).
We wish to derive the electric field $\boldsymbol{E}$ of a dipole.
We start with equation (4.27), which states that
vector $\boldsymbol{E}$ is the negative of the gradient of $\phi$.
\[ \boldsymbol{E} = -\nabla\phi \]

We need a formula for $\phi$, the potential field of an electric dipole.
We make things easier by placing the two point charges along the $z$-axis.
The $+q$ charge is placed $\frac{d}{2}$ units above the $xy$ plane, and
the $-q$ charge is placed $\frac{d}{2}$ units below the $xy$ plane.
We use Equation (4.2) and replace point 1 with point $P(x,y,z)$.
Note that $\phi$ is simply the sum of two potentials,
where each potential is an amount of charge divided by its distance from the orgin.
%% Eq. (6.8)
\begin{equation}
\phi(x,y,z) =
\frac{1}{4\pi\epsilon_0}
\left[
\frac{ q}{\sqrt{x^2 + y^2 + \left[z - \frac{d}{2}\right]^2}} + 
\frac{-q}{\sqrt{x^2 + y^2 + \left[z + \frac{d}{2}\right]^2}}
\right]
\end{equation}

This formula for $\phi$ is too complicated.
We need to simplify the distance expressions in the two denominators.
Our strategy is to make approximations by adding or subtracting second-order $d$ terms.
We can do this because the separation distance $d$ is very small to begin with, 
so $d^2$ terms will be relatively insignificant in comparison.  To start,
we'll apply our strategy to the $[z-\frac{d}{2}]^2$ term in the first denominator.
We expand the square and then discard the $\frac{d^2}{4}$ term.
\begin{eqnarray}
  \left[z-\frac{d}{2}\right]^2  &=&  z^2 - zd + \frac{d^2}{4}  \nonumber\\
  &\approx& z^2 - zd \nonumber
\end{eqnarray}

Now we wish to simplify the distance expression that divides $+q$.
We start with the Pythagorean theorem, subtract $zd$ from both sides,
replace $(z^2 - zd)$ with its approximation $[z-\frac{d}{2}]^2$,
and factor the right side.
\begin{eqnarray*}
 x^2 + y^2 + z^2 &=& r^2 \\
 x^2 + y^2 + z^2 - zd &=& r^2 - zd \\
 x^2 + y^2 + \left[z-\frac{d}{2}\right]^2 &\approx& r^2 \left( 1 - \frac{zd}{r^2} \right)
\end{eqnarray*}

We invert both sides and take the square roots.
We wish to simplify even more, so we let $\frac{zd}{r^2}=a$.
This gives us
\begin{eqnarray*}
  \frac{1}{\sqrt{x^ + y^2 + [z-\frac{d}{2}]^2}}
  &\approx& \frac{1}{\sqrt{r^2 \left( 1 - \frac{zd}{r^2} \right)}} \\
  &\approx& \frac{1}{r \sqrt{1 - a}}
\end{eqnarray*}

Since $a$ is a function of $d$, then any term with $a^2$ will be insignificant.
So we can add $\frac{a^2}{4}$ under the radical to help simplify things.
\begin{eqnarray*}
  &\approx& \frac{1}{r \sqrt{1 - a + \frac{a^2}{4}}} \\
  &\approx& \frac{1}{r \sqrt{\left(1 - \frac{a}{2} \right)^2}} \\
  &\approx& \frac{1}{r \left( 1 - \frac{a}{2} \right)}
\end{eqnarray*}

We now multiply both top and bottom by $(1+\frac{a}{2})$ and simplify the denominator.
\begin{eqnarray*}
  &\approx& \frac{1}{r\left(1-\frac{a}{2}\right)} \cdot \frac{1+\frac{a}{2}}{1+\frac{a}{2}} \\
  &\approx& \frac{1}{r} \cdot \frac{1+\frac{a}{2}}{1-\frac{a^2}{4}}
\end{eqnarray*}

We again add the insignficant $\frac{a^2}{4}$ term to simplify the denominator.
We also replace the $a$ in the numerator with $\frac{zd}{r^2}$.
\begin{eqnarray*}
  &\approx& \frac{1}{r} \cdot \frac{1+\frac{a}{2}}{1-\frac{a^2}{4}+\frac{a^2}{4}} \\
  &\approx& \frac{1}{r} \left( 1 + \frac{1}{2} \; \frac{zd}{r^2} \right) 
\end{eqnarray*}

Finally, we have
\begin{equation*}
  \frac{1}{\sqrt{x^2 + y^2 + [z - \frac{d}{2}]^2}} 
  \approx \frac{1}{r} \left( 1 + \frac{1}{2} \; \frac{zd}{r^2} \right)
\end{equation*}

Similarly, we have
\begin{equation*}
  \frac{1}{\sqrt{x^2 + y^2 + [z + \frac{d}{2}]^2}}
  \approx \frac{1}{r} \left( 1 - \frac{1}{2} \; \frac{zd}{r^2} \right)
\end{equation*}

We are now ready to simplify equation (6.8).
We factor out the two $q$ charges and replace the radicals 
with the approximations derived above.
\begin{eqnarray*}
  \phi(x,y,z)
  &=& \frac{1}{4\pi\epsilon_0}
  \left[ \frac{+q}{\sqrt{x^2 + y^2 + [z - \frac{d}{2}]^2}} + 
    \frac{-q}{\sqrt{x^2 + y^2 + [z + \frac{d}{2}]^2}} \right] \\
  &\approx& \frac{q}{4\pi\epsilon_0}
  \left[ \frac{1}{r} \left(1+\frac{1}{2} \frac{zd}{r^2} \right) -
    \frac{1}{r} \left(1-\frac{1}{2} \frac{zd}{r^2} \right) \right] \\
  &\approx& \frac{q}{4\pi\epsilon_0}
  \left[ \frac{1}{r} + \frac{1}{2} \frac{zd}{r^3} - 
    \frac{1}{r} + \frac{1}{2} \frac{zd}{r^3} \right] \\
  &\approx& \frac{q}{4\pi\epsilon_0} \cdot \frac{zd}{r^3}
\end{eqnarray*}

Rearranging terms we have
%% Eq. (6.9)
\begin{equation}
  \phi(x,y,z) = \frac{1}{4\pi\epsilon_0} \frac{z}{r^3} qd
\end{equation}

We define $p$ as the \emph{dipole moment}.
%% Eq. (6.10)
\begin{equation}
  p = qd
\end{equation}

We replace $qd$ with $p$. We also replace $z/r$ with $\cos\theta$.
%% Eq. (6.11)
\begin{equation}
  \phi(x,y,z) = \frac{1}{4\pi\epsilon_0} \frac{p \cos\theta}{r^2}
\end{equation}

We define $\bv{p}$ as a vector that points from $q_-$ to $q_+$ and has magnitude $p$.
Let $\bv{e}_r$ be the unit radial vector that points from the dipole to point $P$ 
and has magnitude $e_r=1$. (See Figure 6-3).
Recall that $p\;e_r\cos\theta$ is equal to the dot product $\bv{p}\cdot\bv{e}_r$.
Since $e_r=1$ we have
%% Eq. (6.12)
\begin{equation}
  p \cos \theta = \bv{p} \cdot \bv{e}_r
\end{equation}

We can simplify the formula still further.
Let $\bv{r}$ be the vector that points in the same direction as $\bv{e}_r$ but has magnitude $r$.
We can use $\bv{r}$ to modify the $p$ term in Equation (6.11).
\begin{equation*}
  \frac{p\cos\theta}{r^2} = \frac{p\;r\cos\theta}{r^3} = \frac{\bv{p}\cdot\bv{r}}{r^3}
\end{equation*}

We replace $\phi(x,y,z)$ with $\phi(r)$.
This gives us a simplified formula for the potential of an electric dipole.
%% Eq. (6.13)

\hspace{2em} \emph{Dipole potential:}
\vspace{-1.0em}
\begin{equation}
  \phi(r) = \frac{1}{4\pi\epsilon_0} \frac{\bv{p} \cdot \bv{r}}{r^3}
\end{equation}

Our goal is to get a formula for the electric field of a dipole.
We can get it by taking the gradient of $\phi$.
The result is three partial derivatives, which are the three vector components
$(E_x,E_y,E_z)$.
%% nonumber
\begin{eqnarray*}
  \bv{E} &=& - \grad \phi \\
  &=& \left( -\dfdx{\phi}{x}, -\dfdx{\phi}{y}, -\dfdx{\phi}{z} \right)
\end{eqnarray*}

We need to derive the partial derivatives of $\phi$.
We'll use Equation (6.9) and replace $qd$ with $p$.
We'll also let $u=r^2$ so that $u^{3/2}=r^3$.
%% nonumber
\begin{equation*}
  \phi = \frac{p}{4\pi\epsilon_0} z u^{-3/2}
\end{equation*}

Since we have defined $u$, we'll need to use the chain rule for differentiation.
That is, given some function $f(u)$, the derivative of $f$ with respect to $x$ is
%% nonumber
\begin{equation*}
  \dfdx{f}{x} = \dfdx{f}{u} \cdot \dfdx{u}{x}
\end{equation*}

Since $u=r^2$ and $r^2=x^2+y^2+z^2$ we can derive the partial derivatives of $u$
with respect to $x$, $y$, and $z$.
\begin{equation*}
  \dfdx{u}{x} = 2x \qquad \dfdx{u}{y} = 2y \qquad \dfdx{u}{z} = 2z
\end{equation*}

First we derive $E_x$.
The variable $z$ is considered a constant in differentiation with respect to $x$,
so it goes on the outside term along with the constant $p/4\pi\epsilon_0$.
We use the chain rule on the remaining $u^{-3/2}$ term.
The derivatives are taken, the negative signs cancel, the 2's cancel, 
and $u^{-5/2}$ is replaced with $1/r^5$.
\begin{eqnarray*}
  -\dfdx{\phi}{x}
  &=& -\dfdx{}{x} \left( \frac{p}{4\pi\epsilon_0} z u^{-3/2} \right) \\
  &=& - \frac{pz}{4\pi\epsilon_0} \; \dfdx{}{u} \! \left( u^{-3/2} \right) \dfdx{u}{x} \\
  &=& - \frac{pz}{4\pi\epsilon_0} \left[-\frac{3}{2} u^{-5/2}\right] 2x \\
  &=& \frac{p}{4\pi\epsilon_0} \; \frac{3zx}{r^5}
\end{eqnarray*}

Next we derive $E_y$.
The variable $z$ is considered a constant in differentiation with respect to $y$.
The remaining derivation is almost exactly like the one for $x$.
\begin{eqnarray*}
  -\dfdx{\phi}{y}
  &=& -\dfdx{}{y} \left( \frac{p}{4\pi\epsilon_0} z u^{-3/2} \right) \\
  &=& - \frac{pz}{4\pi\epsilon_0} \; \dfdx{}{u} \! \left( u^{-3/2} \right) \dfdx{u}{y} \\
  &=& - \frac{pz}{4\pi\epsilon_0} \left[-\frac{3}{2} u^{-5/2}\right] 2y \\
  &=& \frac{p}{4\pi\epsilon_0} \; \frac{3zy}{r^5}
\end{eqnarray*}

Finally we derive $E_z$.
The variable $z$ is not a constant in differentiation with respect to $z$, 
so it stays inside and is part of the derivative.
We use the rule for the derivative of a product, and then
we apply the chain rule to the second term.
\begin{eqnarray*}
  -\dfdx{\phi}{z}
  &=& -\dfdx{}{z} \left( \frac{p}{4\pi\epsilon_0} z u^{-3/2} \right) \\
  &=& -\frac{p}{4\pi\epsilon_0} \; \dfdx{}{z}\!\left(zu^{-3/2}\right) \\
  &=& -\frac{p}{4\pi\epsilon_0} \left[ \dfdx{z}{z} \cdot u^{-3/2}
      + z \cdot \dfdx{}{z}(u^{-3/2}) \right] \\
  &=& -\frac{p}{4\pi\epsilon_0} \left[ u^{-3/2}
      + z \cdot \dfdx{}{u}(u^{-3/2}) \cdot \dfdx{u}{z} \right] \\
  &=& -\frac{p}{4\pi\epsilon_0} \left[ u^{-3/2} - \frac{3}{2}zu^{-5/2} \cdot 2z \right] \\
  &=& -\frac{p}{4\pi\epsilon_0} \left[ \frac{1}{r^3} -\frac{3z^2}{r^5} \right]
\end{eqnarray*}

Since $z/r=\cos\theta$ we can replace $z^2/r^2$ with $\cos^2\theta$.
This results in a subtraction with common denominators $r^3$.
\begin{eqnarray*}
  &=& -\frac{p}{4\pi\epsilon_0} \left[ \frac{1}{r^3} -\frac{3z^2}{r^2r^3} \right] \\
  &=& -\frac{p}{4\pi\epsilon_0} \left[ \frac{1}{r^3} -\frac{3\cos^2\theta}{r^3} \right] \\
  &=& -\frac{p}{4\pi\epsilon_0} \left[ \frac{1 - 3\cos^2\theta}{r^3} \right]
\end{eqnarray*}

This gives us a simplified formula for $E_z$.
%% Eq. (6.14)
\begin{equation}
  E_z = \frac{p}{4\pi\epsilon_0} \frac{3\cos^2\theta-1}{r^3}
\end{equation}

We combine the two vector components $E_x$ and $E_y$ 
into the \emph{transverse} component $E_{\perp}$, which is perpendicular to the $z$-axis.
\begin{eqnarray*}
  E_{\perp} &=& \sqrt{E_x^2 + E_y^2} \\
  &=& \sqrt{\left(\frac{p}{4\pi\epsilon_0}\frac{3zx}{r^5}\right)^2 
    + \left(\frac{p}{4\pi\epsilon_0}\frac{3zy}{r^5}\right)^2} \\
  &=& \frac{p}{4\pi\epsilon_0}\frac{3z}{r^5}\sqrt{x^2 + y^2} 
\end{eqnarray*}

We split these three terms into four terms and apply two trigonometric identities.
Recall that $\theta$ is the angle between the z-axis and vector $\bv{r}$.
\begin{eqnarray*}
  &=& \frac{p}{4\pi\epsilon_0} \cdot \frac{3}{r^3} \cdot \frac{z}{r}
    \cdot \frac{\sqrt{x^2 + y^2}}{r} \\
  &=& \frac{p}{4\pi\epsilon_0} \cdot \frac{3}{r^3} \cdot \cos\theta \cdot \sin\theta
\end{eqnarray*}

This gives us a simplified formula for $E_{\perp}$.
%% Eq. (6.15)
\begin{equation}
  E_{\perp} = \frac{p}{4\pi\epsilon_0} \frac{3 \cos\theta \sin\theta}{r^3}
\end{equation}

Figure 6-4 illustrates the electric field of a dipole.
Note the vector $\boldsymbol{E}$ and its two orthogonal components $E_z$ and $E_{\perp}$.
Both of these have been derived in terms of the dipole moment $p=qd$,
trigonometric functions on the angle $\theta$,
and the distance $r$ from the field point to the center of the dipole.
The electric field $\boldsymbol{E}$ of the dipole
varies inversely as the cube of the distance from the dipole.
The magnitude of the field is
%% nonumber
\begin{equation*}
  E = \sqrt{E_z^2 + E_{\perp}^2}
\end{equation*}

