%% 1-4.tex
\section{The laws of electromagnetism}

Here are four laws of electromagnetism.
They are expressed using the two ideas, flux and circulation,
in conjunction with the two fields, $\bv{E}$ and $\bv{B}$.

The first law describes the flux of $\bv{E}$.
Given some arbitrary closed surface $S'$ (e.g. a spherical membrane) 
and some charges in the space bounded by that surface,
the flux of electric field $\bv{E}$ through closed surface $S'$ is
the net charge inside divided by $\epsilon_0$ (a convenient constant).
%% Eq. (1.6)
\begin{equation}
  \mathrm{Flux\ of\ } \bv{E} \mathrm{\ thru\ } S' = 
  \frac{\mathrm{net\ charge\ inside}}{\epsilon_0}
\end{equation}

The second law describes the circulation of $\bv{E}$.
Given some arbitrary open surface $S$ (e.g. a thin disc) whose edge is the curve $C$,
the circulation of electric field $\bv{E}$ around curve $C$ is equal to the negative of
the time derivative of the flux of magnetic field $\bv{B}$ through open surface $S$.
%% Eq. (1.7)
\begin{equation}
  \mathrm{Circ\ of\ } \bv{E} \mathrm{\ around\ } C =
  -\frac{d}{dt} \Big( \mathrm{flux\ of\ } \bv{B} \mathrm{\ thru\ } S \Big)
\end{equation}

\newpage

The third law describes the flux of $\bv{B}$.
Given some arbitrary closed surface $S'$
the flux of magnetic field $\bv{B}$ through closed surface $S'$ is zero.
There are no magnetic ``charges'' inside the surface.
\begin{equation}
  \mathrm{Flux\ of\ } \bv{B} \mathrm{\ thru\ } S' = 0
\end{equation}

The fourth law describes the circulation of $\bv{B}$.
Given some arbitrary open surface $S$ whose edge is the curve $C$,
the circulation of magnetic field $\bv{B}$ around curve $C$ is proportional to 
the time derivative of the flux of electric field $\bv{E}$ through open surface $S$, plus
the flux of the electric current $I$ through open surface $S$, divided by $\epsilon_0$.
The constant $c^2$ is the square of the velocity of light.
It appears because magnetism is in reality a relativistic effect of electricity.
\begin{eqnarray}
  c^2 (\mathrm{Circ\ of\ } \bv{B} \mathrm{\ around\ } C)
  &=& \frac{d}{dt} \Big( \mathrm{flux\ of\ } \bv{E} \mathrm{\ thru\ } S \Big) \nonumber\\
  & & \quad  + \; \frac{\mathrm{flux\ of\ } I \mathrm{\ thru\ } S}{\epsilon_0}
\end{eqnarray}

The most remarkable consequence of our equations is that 
the combination of Eq. (1.7) and Eq. (1.9) contains the explanation of
the radiation of electromagnetic effects over large distances.
Consider an antenna element: a current changes and
causes the magnetic field around the element to change.
By \mbox{Eq. (1.7)} the changing magnetic field causes a changing electric field.
And by \mbox{Eq. (1.9)} the changing electric field causes a changing magnetic field.
This interchange continues indefinitely.
The electric and magnetic fields work their way through space
without the need of charges or currents.


\begin{comment}
%%%%%%%%%%%%%%%%%%%%%%%%%%%%%%%%%%%%%%%%%%%%%%%%%%%%%%%%%%%%%%%%%%%%%%%%%%%%%%%%

We can verify that equations (1.6), (1.7), and (1.9) produce the correct units.
In equation (1.6), the ``net charge inside'' is in coulombs.
The constant $e_0$ is defined in equation (4.10).
It's value is approximately $8.854 \times 10^{-12}$ coul$^2$/nt$\cdot$mt$^2$.
One of the coul terms cancels in the division, and the result is the units for electric flux.
\begin{equation*}
  \frac{coul}{\left( \frac{coul^2}{nt \cdot mt^2} \right)}
  = \frac{nt \cdot mt^2}{coul}
  \equiv \mathrm{Electric\ flux}
\end{equation*}

In equation (1.7), the time derivative of the units for magnetic flux
has the effect of dividing by seconds.
The result is the units for electric circulation.
\begin{equation*}
  \frac{d}{dt} \left( \frac{nt \cdot mt \cdot sec}{coul} \right)
  = \frac{nt \cdot mt}{coul}
  \equiv \mathrm{Electric\ circ.}
\end{equation*}

In equation (1.9), we use only the first term, and then we divide both sides by $c^2$.
The time derivative of the units for electric flux has the effect of dividing by seconds.
The two mt$^2$ terms cancel each other in the division.
One of the sec terms also cancels.
The result is the units for magnetic circulation.
\begin{equation*}
  \frac{\frac{d}{dt} \left(\frac{nt \cdot mt^2}{coul} \right)}
  {\frac{mt^2}{sec^2}} =
  \frac{\frac{nt \cdot mt^2}{coul \cdot sec}}
  {\frac{mt^2}{sec^2}} =
  \frac{nt \cdot sec}{coul}
  \equiv \mathrm{Magnetic\ circ.}
\end{equation*}

%%%%%%%%%%%%%%%%%%%%%%%%%%%%%%%%%%%%%%%%%%%%%%%%%%%%%%%%%%%%%%%%%%%%%%%%%%%%%%%%
\end{comment}

