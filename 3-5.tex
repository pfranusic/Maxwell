%% 3-5.tex
\section{The circulation of a vector field}

If $\bv{C}$ is any vector field, we take its component along a curved line
and take the integral of this component all the way around a complete loop.
The integral is called the \emph{circulation} of the vector field around the loop.

Consider Figure 3-7 with vector field $\bv{C}$ and closed loop $\Gamma$.
The tangential component of $\bv{C}$ at any point along the loop is $C_t$.
The closed loop consists of an infinite number of infinitesimal vector segments 
$d\bv{s}=(ds_x,ds_y,ds_z)$.
The magnitude of $d\bv{s}$ at any point along the loop is $ds$.

Given the two vectors $\bv{C}$ and $d\bv{s}$, their dot product is $|\bv{C}||d\bv{s}|\cos\theta$.
Now the tangential component $C_t = |\bv{C}|\cos\theta$.
And the magnitude $ds=|d\bv{s}|$.
Therefore we have the following identity for the integral all the way around $\Gamma$.
The circle on the integral sign means that the integral is taken around the entire loop.
%% Eq. (3.30)
\begin{equation}
  \oint_\Gamma C_t \; ds = \oint_\Gamma \bv{C} \cdot d\bv{s}
\end{equation}

Consider Figure 3-8.
We split $\Gamma$ into two smaller loops $\Gamma_1$ and $\Gamma_2$.
Both of these smaller loops have the segment $\Gamma_{ab}$ in common.
However, the directions of $d\bv{s}_1$ and $d\bv{s}_2$ are opposite each other.
When we compute the circulations around the smaller loops and then add them together,
the two partial circulations for $\Gamma_{ab}$ cancel each other,
so that the sum is the same as the circulation around the big loop $\Gamma$.

Now consider Figure 3-9.
Here we have some arbitrary surface bounded by a closed loop $\Gamma$.
We subdivide the surface into an infinite number of infinitesimal squares.
And since the squares are so small, we may make the assumption that each square is flat, 
even though the surface itself may not be flat at all.
So now we can calculate the circulation around $\Gamma$.
It is the sum of the circulations around each of the infinitesimal squares.

