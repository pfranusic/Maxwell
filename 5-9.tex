%% 5-9.tex
\section{The fields of a conductor}

We consider the electric field at the surface
of a conductor with no continuous source of current.
In this ``electrostatic'' situation, all charges reside on the \emph{surface},
so the charge density \emph{inside} the conductor is zero.
The electric field \emph{just outside} the surface is normal to the surface.

Figure 5-11 shows the cross section of a \textsc{conductor} with no continuous source of current.
Positive charges reside on the surface with a \textsc{local surface charge density} of $\sigma$ 
unit charges per unit area.  The electric field inside the conductor is zero.
In order to determine the electric flux produced by the positive surface charges,
a small cylindrical box is used as the \textsc{gaussian surface}.

The two circular surfaces of the box each have area $A$,
but only the outside surface contributes to the electric flux.
The flux magnitude through this outside surface is $EA$.
The charge on this outside surface is $\sigma A$.
We use Gauss' law to derive the magnitude of the electric field.

\emph{Outside a conductor:}
\vspace{-1.5em}

%% Eq. (5.8)
\begin{equation}
  E = \frac{\sigma}{\epsilon_0}
\end{equation}

