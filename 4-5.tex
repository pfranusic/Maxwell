%% 4-5.tex
\section{The flux of $\bv{E}$}

What is the flux of $\bv{E}$ out of an arbitrary closed surface
in the neighborhood of a point charge?
We first take the simple surface shown in Figure 4-5 where the charge lies on the outside.
This surface has four radial faces which are parallel to $\bv{E}$.
It also has two spherical faces $a$ and $b$ which are normal to $\bv{E}$.
The area of $a$ is less than the area of $b$ by a factor of $r^2$.
The magnitude $E_a$ is greater than magnitude $E_b$ by a factor of $1/r^2$.
Therefore the flux of $\bv{E}$ into face $a$ is cancelled by the flux out of face $b$.
\textcolor{red}{The following equation doesn't quite correlate with the previous text.}
%% Eq. (4.30)
\begin{equation}
  \int_S E_n \; da = 0
\end{equation}

Next we take the surface shown in Figure 4-6.
It is called an \emph{infinitesimal truncated cone}.
It's similar to the previous surface in that it has four radial faces.
But it's different in two ways. First, it is infinitesimally narrow. Second, 
it has two flat tilted faces instead of two spherical faces. That is, each face 
is perfectly flat, and each face is tilted from the normal by some angle $\theta$.
The area of each face, $\Delta a$, increases by a factor of $1/\cos\theta$ as $\theta$ is increased.
And the magnitude of the normal component, $E_n$, decreases by a factor of $\cos\theta$
as $\theta$ is increased.
Therefore the product $E_n \Delta a$ is the same regardless of the angle $\theta$.
And the flux of $\bv{E}$ out of this surface is zero.

Of course, \emph{any} surface can be subdivided into an infinite number of
infinitesimal truncated cones.
Figure 4-7 shows an arbitrary cylindrical surface with a few truncated cones.
Now we can say that the total flux of $\bv{E}$ out of \emph{any} surface $S$ is zero
for the field of a point charge that lies outside $S$.

But what if the point charge lies inside $S$?
How do we compute the flux of $\bv{E}$ through $S$?
We could try using pairs of truncated cones as shown in Figure 4-8.
But the two partial fluxes will add rather than cancel.
The flux out of a surface that \emph{surrounds} a charge is \emph{not} zero.

We use a ``trick'' to compute the flux of $\bv{E}$ 
out of a surface $S$ that surrounds a point charge.
We \emph{isolate} the charge by surrounding it with another surface $S'$
that is much smaller and also spherical.
So now we have a volume bounded by the two surfaces $S$ and $S'$.
The charge is outside this volume, so the flux of $\bv{E}$ out of this volume is zero.

But what is the flux of $\bv{E}$ through surface $S'$?
The radius of our little sphere is $r$, and from Equation (4.11),
we have the value of $\bv{E}$ everywhere on $S'$ and normal to $S'$.
%% nonumber
\begin{equation*}
  \frac{1}{4\pi\epsilon_0} \; \frac{q}{r^2}
\end{equation*}

To get the total flux of $\bv{E}$ through the entire surface $S'$
we simply multiply by the surface area of $S'$.
The $r^2$ terms cancel, and we get an expression that is independent of the radius.
%% Eq. (4.31)
\begin{equation}
  \mathrm{Flux\ through\ } S' =
  \left( \frac{1}{4\pi\epsilon_0} \; \frac{q}{r^2} \right)
  \left( 4 \pi r^2 \right) = \frac{q}{\epsilon_0}
\end{equation}

So we conclude that the flux of $\bv{E}$ through a surface $S$
depends on the location of the point charge, 
whether it is outside of $S$ or inside of $S$.
%% Eq. (4.32)
\begin{equation}
  \int\limits_{\mathrm{any\ surface\ }S} E_n \; da = 
  \left\{ \begin{array}{c}
    0 \qquad \mathrm{if\ } q \mathrm{\ is\ outside\ of\ } S \\
    q/\epsilon_0 \qquad \mathrm{if\ } q \mathrm{\ is\ inside\ of\ } S \\
  \end{array} \right.
\end{equation}

