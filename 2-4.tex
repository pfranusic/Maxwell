%% 2-4.tex
\section{The operator $\grad$}

Now we can do something that is extremely amusing and ingenious ---
and characteristic of the things that make mathematics beautiful.
We can abstract the gradient $\grad$ away from the $T$.
We take the $T$ out of equation (2.26) and leave the operators
``hungry for something to differentiate.''
% Eq. (2.27)
\begin{equation}
  \dfdx{}{x} = \dfdx{}{x'} \cos \theta - \dfdx{}{y'} \sin \theta
\end{equation}

We make $\grad$ a \emph{vector operator}.
Just as the vector $\grad T$ in equation (2.14) has three components,
so does the vector operator $\grad$.
% Eq. (2.28)
\begin{equation}
  \grad = \Bigg( \dfdx{}{x}, \dfdx{}{y}, \dfdx{}{z} \Bigg)
\end{equation}

Another way of writing the vector operator $\grad$ is 
$(\nabla_x,\nabla_y,\nabla_z)$ where
% Eq. (2.29)
\begin{equation}
  \nabla_x = \dfdx{}{x} \qquad
  \nabla_y = \dfdx{}{y} \qquad
  \nabla_z = \dfdx{}{z}
\end{equation}

We need to remember that the operator $\grad$ always precedes a scalar variable.
(For example, $\grad T$). 
An expression with $\grad$ on the right side of the scalar variable is meaningless.
For example, $T \grad$ has an $x$-component that is not a number.
% Eq. (2.30)
\begin{equation}
  T \dfdx{}{x}
\end{equation}

What is to be differentiated must be placed on the right of the $\grad$.
The commutative law for multiplication does not apply for $\grad$.
However, if we have a scalar $T$ and a vector $\bv{A}$ 
we can represent their product either way.
% Eq. (2.31)
\begin{equation}
  T \bv{A} = \bv{A} T
\end{equation}

What is the direction of the vector $\grad T$?
On a three-dimensional graph,
it's the direction of the steepest uphill slope of $T(x,y,z)$.

