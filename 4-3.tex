%% 4-3.tex
\section{Electric potential}

This section develops the idea of electric potential $/phi$.
We begin by specifying the work done to move a charge some finite distance.
Figure 4-2 shows a charge $q$ moving on a path from point $a$ to point $b$.
There is some distribution of charge, so there's an electric field.
At each point along the path, the charge feels an electrical force $\bv{F}$,
and the charge makes an infinitesimal move in the direction of $d\bv{s}$.
The work $W$ done \emph{against} the electrical forces to carry $q$ along this path
is the \emph{negative} of the component of the electrical force in the 
direction of motion, integrated along the path from $a$ to $b$.
%% unnumbered equation before (4.19)
\begin{equation*}
  W = - \int_a^b \bv{F} \cdot d\bv{s}
\end{equation*}

We divide both sides by $q$ to get the work done to carry \emph{one unit} of charge.
Since $\bv{F} /q = \bv{E}$ we have
%% Eq. (4.19)
\begin{equation}
  W \mathrm{(unit)} = - \int_a^b \bv{E} \cdot d\bv{s}
\end{equation}

In this next equation, there are three expressions: left, middle, right.
The left expression is the right side of Eq. (4.19).
I know how to get the right expression from the middle expresssion.
\textcolor{red}{But how do we get the middle expression from the left expression?}
I think we need to refer to Coulomb's law as shown in Eq. (4.9).
Another thing: \textcolor{red}{Why is no work done from $a$ at $a'$?}
The text explains that the field is at right angle to the direction of motion.
But I need something a little more substantial.
%% Eq. (4.20)
\begin{equation}
  - \int_a^b \bv{E} \cdot d\bv{s}
  = \frac{q}{4\pi\epsilon_0} \int_{a'}^b \frac{dr}{r^2}
  = - \frac{q}{4\pi\epsilon_0} \Big(\frac{1}{r_a} - \frac{1}{r_b}\Big)
\end{equation}

The idea in Figure 4-3 is that no matter \emph{which} path we choose,
the work to get from $a$ to $b$ will always be the same.
%% unnumbered equation before (4.21)
\begin{equation*}
  \left. \begin{array}{l}
    W \mathrm{(unit)} \\
    a \to b
  \end{array} \right\}
  = - \int_a^b \bv{E} \cdot d\bv{s}
\end{equation*}

Now we introduce the electric potential $\phi$.
Figure 4-4 shows the two points $a$ and $b$ plus some arbitrary reference point $P_0$.
We define $\phi(a)$ as the work required to move a charge from $P_0$ to $a$,
and $\phi(b)$ as the work required to move a charge from $P_0$ to $b$.
Now we want the work required to move the charge from $a$ to $P_0$ 
and then from $P_0$ to $b$. This is $-\phi(a)+\phi(b)$.
%% Eq. (4.21)
\begin{equation}
  - \int_a^b \bv{E} \cdot d\bv{s} = \phi(b) - \phi(a)
\end{equation}

Note that this looks a lot like the Fundamental Theorem of Calculus, 
except for the minus sign.  So now, given some arbitrary reference point $P_0$, 
we can define the electrostatic potential function for any point $P$, where $\phi(P_0)=0$.
It is a scalar function of $P=(x,y,z)$.
%% Eq. (4.22)

\hspace{2em}\emph{Electrostatic potential:}
\begin{equation}
  \phi(P) = - \int_{P_0}^P \bv{E} \cdot d\bv{s}
\end{equation}

We replace the left side with $\phi(x,y,z)$.
We replace the right side with the right-most expression from Eq. (4.20),
except that we replace $r_b$ with $r$, and
$r_a$ with $r_0$, the distance of q from $P_0$.
\begin{equation*}
  \phi(x,y,z) = - \frac{q}{4\pi\epsilon_0} \bigg(\frac{1}{r_0}-\frac{1}{r}\bigg)
\end{equation*}

We'd like to get rid of the $1/r_0$ term.  We can do this if
we put the reference point $P_0$ \emph{really far away} and then take the limit
as $r_0$ approaches infinity.
\begin{equation*}
  \lim_{r_0 \to \infty} -\frac{q}{4\pi\epsilon_0} \bigg(\frac{1}{r_0}-\frac{1}{r}\bigg)
  = \frac{q}{4\pi\epsilon_0} \; \frac{1}{r}
\end{equation*}

With the reference point at infinity, 
the $1/r_0$ term becomes a 0, the two minus signs cancel,
and we get a nice clean equation for the electrostatic potential
at any point $(x,y,z)$ where a single charge $q$ is at the orgin $(0,0,0)$.
\textcolor{red}{But what is $r$? Is it $(x^2 + y^2 + z^2)^{1/2}$?}
%% Eq. (4.23)
\begin{equation}
  \phi(x,y,z) = \frac{q}{4\pi\epsilon_0} \; \frac{1}{r}
\end{equation}

\textcolor{red}{
The last two equations in this section are given without derivation.
Even though they seem to make sense, this still bothers me.
I'd like to see them derived.}

This equation gives the potential at some point (1) in an electric field
caused by a group of charges.
It looks a lot like Eq. (4.13), except it's missing an $\bv{e_{1j}}/r_{1j}$ term.
\textcolor{red}{So how do we get from Eq. (4.13) to Eq. (4.24)?}
%% Eq. (4.24)
\begin{equation}
  \phi(1) = \sum_j \frac{1}{4\pi\epsilon_0} \; \frac{q_j}{r_{1j}}
\end{equation}

This equation gives the potential at some point (1) in an electric field
caused by a distribution of charges.
It looks a lot like Eq. (4.16), except it's missing an $\bv{e_{12}}/r_{12}$ term.
\textcolor{red}{So how do we get from Eq. (4.16) to Eq. (4.25)?}
%% Eq. (4.25)
\begin{equation}
  \phi(1) = \frac{1}{4\pi\epsilon_0} \int  \frac{\rho(2)\;dV_2}{r_{12}}
\end{equation}

