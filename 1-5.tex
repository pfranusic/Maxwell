%% 1-5.tex
\section{What are the fields?}

All this business of fluxes and circulations is pretty abstract.
What is \emph{actually} happening?
Field lines are helpful, but they do not contain the deepest principle 
of electrodynamics, which is the superposition principle.
So what is the \emph{most convenient} way to look at electrical effects?
The best way is to use the abstract field idea.
That it is abstract is unfortunate but necessary.

What happens when two charges in space move parallel to each other at the same speed?
A stationary observer will observe a magnetic field surrounding the two charges.
But an observer riding along with the charges will observe no magnetic field.
So magnetism is really a relativistic effect.

Consider the apparatus in Figure 1-8.
Two parallel wires each carry a current.
When the currents are in the same direction, the two wires attract.
The average speed of the electrons in the wires is about 0.01 centimeters per second,
so $v^2/c^2$ is about $10^{-25}$.
In the Lorentz force equation,
the term for the electrical force disappears because of the almost perfect balance --- 
the wires have the same number of protons as electrons.
The small relativistic term which we call the magnetic force is the only term left.
It becomes the dominant term.

