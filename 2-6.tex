%% 2-6.tex
\section{The differential equation of heat flow}

We can use vector notation to describe all sorts of elementary physics relations.
In this section we describe heat flow using vector notation.
We are given the slab of material in Figure 2-7(a).
$A$ is the area of the large face.
$d$ is the thickness of the slab.
$T_2$ is the temperature on the hot face.
$T_1$ is the area on the cold face.
$J$ is the thermal energy that passes through the slab per unit time.
$\kappa$ is the \emph{thermal conductivity} constant.
%% Eq. (2.42)
\begin{equation}
  J = \kappa \big( T_2 - T_1 \big) \frac{A}{d}
\end{equation}

Now we complicate things.
The slab becomes odd-shaped and the temperature within the slab
varies in peculiar ways.
Our strategy is to describe infinitesimal differences.
We take an infinitesimal piece of the slab, as shown in Figure 2-7(b).
The two parallel curves are isothermals.
$\Delta A$ is the area of the small face (not shown in the figure).
$\Delta s$ is the distance between the isothermals.
$\Delta T$ is the difference in temperature between the isothermals.
$\kappa$ is the same thermal conductivity constant as before.
%% Eq. (2.43)
\begin{equation}
  \Delta J = \kappa \Delta T \frac{\Delta A}{\Delta s}
\end{equation}

\newpage
Now we transform the scalar equation (2.43) to a vector equation (2.44).
This is explained using magnitudes.
The first step is to divide both sides by $\Delta A$ and rearrange.
This gets rid of the $\Delta A$ on the right side.
\begin{equation*}
  \frac{\Delta J}{\Delta A} = \kappa \frac{\Delta T}{\Delta s}
\end{equation*}

The first transformation is from scalar $\Delta J / \Delta A$ to vector $\bv{h}$.
In equation (2.9), $\Delta J / \Delta A$ was defined as the magnitude of $\bv{h}$.
The direction of $\bv{h}$ is perpendicular to the isothermals.
\begin{equation*}
  \frac{\Delta J}{\Delta A} = | \bv{h} |
\end{equation*}

The other transformation is from scalar $\Delta T / \Delta s$ to vector $\grad T$.
$\Delta T / \Delta s$ is the rate of change of $T$ with position.
It's the maximum rate of change, and therefore the magnitude of $\grad T$.
\begin{equation*}
% \frac{\Delta T}{\Delta s} \stackrel{?}{\to} \grad T
  \frac{\Delta T}{\Delta s} = | \grad T |
\end{equation*}

We substitute these two transformations into Equation (2.43) and then drop the magnitude bars.
The direction of $\grad T$ is opposite that of $\bv{h}$, so there's a sign change.
%% Eq. (2.44)
\begin{equation}
  \bv{h} = - \kappa \grad T
\end{equation}

