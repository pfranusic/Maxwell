%% 5-10.tex
\section{The field in a cavity of a conductor}

What is the electric field inside an empty cavity within a conductor?
Figure 5-12 shows the cross section of a conductor containing an empty cavity.
There is no continuous source of current in the conductor, and
positive charges reside on the outside surface of the conductor.
We use a false hypothesis and Stokes's Theorem (3.38) to show that 
an electric field cannot exist inside the empty cavity.

The hypothesis is that the surface of the cavity contains 
a cluster of positive charges on one side and a cluster of negative charges on the other.
Some arbitrary loop $\Gamma$ crosses the cavity
between these two clusters and returns through the conductor.
We now compute the integral of the electric field $\bv{E}$ around the loop $\Gamma$.
Inside the cavity, the integral along $\Gamma$ is nonzero,
because the positive and negative charges produce an electric field.
On the return trip, the integral along $\Gamma$ is zero,
because $\bv{E}=0$ everywhere inside the conductor.
Therefore the integral around the entire loop must be nonzero.
%% nonumber
\begin{equation*}
  \oint \bv{E} \cdot d\bv{s} \not = 0 \; ???
\end{equation*}

Stokes' Theorem says this integral is equal to the integral of an infinite number of
infinitesimal $\grad \times \bv{E}$ circulations in the surface bounded by $\Gamma$.
But $\grad \times \bv{E} = 0$ in electrostatics.
So the total integral of the circulations is zero, which means the 
integral around $\Gamma$ is also zero.
This contradicts the hypothesis, so the hypothesis must be false.
There are no clusters of positive and negative charges on the surface of the cavity.
As such, no electric fields can exist inside the cavity.

