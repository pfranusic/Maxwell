%% 5-1.tex
\section{Electrostatics in Gauss's law plus \ldots}

There are two laws of electrostatics.
All the predictions of electrostatics follow from these two laws.
\begin{enumerate}

\item
  \emph{Gauss's law} says that the flux of the electric field $\bv{E}$ through the closed 
  surface $S$ of any volume is proportional to the internal charge $Q_{\mathrm{int}}$.
  \begin{equation*}
    \int\limits_{\genfrac{}{}{0pt}{1}{\mathrm{any\ closed}}{\mathrm{surface\ }S}}
    \bv{E} \cdot \bv{n} \; da = \frac{Q_{\mathrm{int}}}{\epsilon_0}
  \end{equation*}

\item
  \emph{$\bv{E}$ is a gradient}, and therefore,
  the circulation of the electric field is zero.
  \begin{eqnarray*}
    \bv{E} &=& - \grad \phi \\
    \grad \times \bv{E} &=& \grad \times (-\grad\phi) \ = \ -(\grad\times\grad)\phi \ = \ 0
  \end{eqnarray*}

\end{enumerate}

In this chapter we will tackle several problems that can be easily solved using these two laws.
But each solution will require a little ingenuity.
