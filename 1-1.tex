%% 1-1.tex
\section{Electrical forces}

Consider a force like gravitation which varies 
predominantly inversely as the square of the distance, but which is 
about a \emph{billion-billion-billion-billion} times stronger.
And with another difference.  There are two kinds of \emph{charges},
which we call positive and negative, where like charges repel and unlike charges attract.
There is such a force: the electrical force.
Like the gravitational force, the electrical force $F$ decreases inversely
as the square of the distance $r$ between two charges $q_1$ and $q_2$.
This relationship is called Coulomb's law.
\begin{equation*}
  F = \frac{1}{4 \pi \epsilon_0} \frac{q_1 q_2}{r^2}
\end{equation*}

But Coulomb's law is not precisely true when charges are moving.
The electrical forces depend also on the motions of the charges in a complicated way.
One part of the force between moving charges we call the \emph{magnetic} force,
which is really one aspect of an electrical effect.
This is why we call the subject ``electromagnetism.''

We can write the electromagnetic force on a charge $q$ in a simple way.
We use two vectors, $\bv{E}$ and $\bv{B}$, 
where $\bv{E}$ is the \emph{electric field} at the location of $q$,
and $\bv{B}$ is the \emph{magnetic field} at the location of $q$.
The electrical forces from all the other charges in the universe
can be summarized by giving just these two vectors.
The Lorentz force $\bv{F}$ on a charge $q$ moving with a velocity $\bv{v}$ is
%% Eq. (1.1)
\begin{equation}
  \bv{F} = q(\bv{E} + \bv{v} \times \bv{B})
\end{equation}

\newpage

From mechanics, we know how to calculate the motion of a particle, i.e., $\bv{F} = m \bv{a}$.
The equation for the Lorentz force can be combined with the equation of motion so that
if $\bv{E}$ and $\bv{B}$ are given, we can find the motions.
\begin{equation}
  \frac{d}{dt} \Bigg[ \frac{m \bv{v}}{\sqrt{1 - v^2 / c^2}} \Bigg]
  = q(\bv{E} + \bv{v} \times \bv{B})
\end{equation}

An important simplifying principle is the \emph{principle of superposition} of fields.
If one set of charges produces field $\bv{E}_1$ and another set produces field $\bv{E}_2$
then the superposition $\bv{E}$ of these two fields is simply their sum.
The principle of superposition holds for both electric fields and magnetic fields.
\begin{eqnarray}
  \bv{E} &=& \bv{E}_1 + \bv{E}_2 \\
  \bv{B} &=& \bv{B}_1 + \bv{B}_2 \nonumber
\end{eqnarray}

