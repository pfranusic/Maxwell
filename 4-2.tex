%% 4-2.tex
\section{Coulomb's law; superposition}

Coulomb's law says that --- between two charges at rest ---
there is a force directly proportional to the product of the charges,
and inversely proportional to the square of the distance between them.
This force is along a straight line through the two charges.
$\bv{F}_1$ is the force on $q_1$.
$\bv{F}_2$ is the force on $q_2$.
$\bv{e}_{12}$ is the unit vector that points from $q_2$ to $q_1$.
$r_{12}$ is the distance between $q_1$ and $q_2$.

%% Eq. (4.9)
\hspace{2em}\emph{Coulomb's law:}
\vspace{-0.5em}
\begin{equation}
  \bv{F}_1 = \frac{1}{4\pi\epsilon_0} \; \frac{q_1 q_2}{r^2_{12}} \; \bv{e}_{12} = - \bv{F}_2
\end{equation}

In the mks system, the constant of proportionality $1/4\pi\epsilon_0$ 
is defined as exactly $10^{-7}$ times the speed of light squared.
The speed of light $c$ is approximately $3 \times 10^8$ meters per second.
A volt has units kilogram meters squared per coulomb seconds squared.
Therefore $1/4\pi\epsilon_0$ is approximately 9 Giga volt meters per coulomb.
%% Eq. (4.10)
\begin{eqnarray}
  \frac{1}{4\pi\epsilon_0} &=& 
  \Big( 10^{-7} \; \textstyle{\frac{\mathrm{kg} \cdot \mathrm{mt}}{\mathrm{coul}^2}} \Big)
  \Big( c \; \textstyle{\frac{\mathrm{mt}}{\mathrm{sec}}} \Big)^2 \nonumber\\
  &\approx& 9.0 \times 10^9 \; \textstyle{\frac{\mathrm{volt} \cdot \mathrm{mt}}{\mathrm{coul}}}
\end{eqnarray}

When there are more than two charges present,
the \emph{principle of superposition} says that the force on any one charge
is the vector sum of the Coulomb forces from each of the other charges.

We now introduce the idea of an electric field $\bv{E}$.
Consider equation (4.9), where $\bv{F}_1$ is the force on $q_1$.
Let $\bv{E}(1)$ be the force per unit charge on $q_1$ due to $q_2$.
In other words, let $\bv{E}(1) = \bv{F}_1 / q_1$.
We also say that ``$\bv{E}(1)$ is the electric field at the point (1).''
%% Eq. (4.11)
\begin{equation}
  \bv{E}(1) = \frac{1}{4\pi\epsilon_0} \; \frac{q_2}{r^2_{12}} \; \bv{e}_{12}
\end{equation}

Let $\bv{r}_{12}$ be the displacement vector from point (2) to point (1).
Then $r_{12}$ is the distance scalar between point (2) and point (1).
The unit vector $\bv{e}_{12}$ can be written in terms of $\bv{r}_{12}$.
This allows us to rewrite Eq. (4.11) in terms of $\bv{r}_{12}$.
\begin{eqnarray*}
  \bv{r}_{12} &=& \big((x_1 - x_2), (y_1 - y_2), (z_1 - z_2)\big) \\
  r_{12} &=& \sqrt{(x_1 - x_2)^2 + (y_1 - y_2)^2 + (z_1 - z_2)^2} \\
  \bv{e}_{12} &=& \frac{\bv{r}_{12}}{|\bv{r}_{12}|} = \frac{\bv{r}_{12}}{r_{12}} \\
  \bv{E}(1) &=& \frac{1}{4\pi\epsilon_0} \; \frac{q_2}{r^3_{12}} \; \bv{r}_{12}
\end{eqnarray*}

Of course, $\bv{E}(1)$ is a vector, so we really have three equations ---
one for $E_x(1)$, one for $E_y(1)$, and one for $E_z(1)$.
Remember that the symbol ``(1)'' expands to $(x_1,y_1,z_1)$ and 
that vector $\bv{r}_{12}$ has the three components
$(x_1 - x_2)$, $(y_1 - y_2)$, and $(z_1 - z_2)$.
%% Eq. (4.12)
\begin{eqnarray}
  E_x (x_1,y_1,z_1) &=& \frac{q_2}{4\pi\epsilon_0} \; \frac{x_1 - x_2}
  {[(x_1 - x_2)^2 + (y_1 - y_2)^2 + (z_1 - z_2)^2]^{3/2}} \qquad \\
  E_y (x_1,y_1,z_1) &=& \frac{q_2}{4\pi\epsilon_0} \; \frac{y_1 - y_2}
  {[(x_1 - x_2)^2 + (y_1 - y_2)^2 + (z_1 - z_2)^2]^{3/2}} \nonumber\\
  E_z (x_1,y_1,z_1) &=& \frac{q_2}{4\pi\epsilon_0} \; \frac{z_1 - z_2}
  {[(x_1 - x_2)^2 + (y_1 - y_2)^2 + (z_1 - z_2)^2]^{3/2}} \nonumber
\end{eqnarray}

Now we expand the situation to include many charges, not just two.
Let $q_j$ be the magnitude of the $j$th charge.
Let $\bv{r}_{1j}$ be the displacement vector from $q_j$ to $q_1$.
Then $r_{1j}$ is the distance scalar between $q_j$ and $q_1$.
The principle of superposition allows us to simply add the forces.
%% Eq. (4.13)
\begin{equation}
  \bv{E}(1) = \sum_j \frac{1}{4\pi\epsilon_0} \; \frac{q_j}{r^2_{1j}} \; \bv{e}_{1j}
\end{equation}

Again, since $\bv{E}(1)$ is a vector, we have three equations.
Note that $\bv{e}_{1j}$ is replaced with $\bv{r}_{1j} / r_{1j}$,
where $\bv{r}_{1j}$ has the three components $(x_1 - x_j)$, $(y_1 - y_j)$, and $(z_1 - z_j)$.
%% Eq. (4.14)
\begin{eqnarray}
  E_x (x_1,y_1,z_1) &=& \sum_j \frac{q_j}{4\pi\epsilon_0} \; \frac{(x_1 - x_j)}
  {[(x_1 - x_j)^2 + (y_1 - y_j)^2 + (z_1 - z_j)^2]^{3/2}} \qquad \\
  E_y (x_1,y_1,z_1) &=& \sum_j \frac{q_j}{4\pi\epsilon_0} \; \frac{(y_1 - y_j)}
  {[(x_1 - x_j)^2 + (y_1 - y_j)^2 + (z_1 - z_j)^2]^{3/2}} \nonumber\\
  E_z (x_1,y_1,z_1) &=& \sum_j \frac{q_j}{4\pi\epsilon_0} \; \frac{(z_1 - z_j)}
  {[(x_1 - x_j)^2 + (y_1 - y_j)^2 + (z_1 - z_j)^2]^{3/2}} \nonumber
\end{eqnarray}

Consider the situation where we have a collection of point charges 
that are spread out from each other in space.
We may approximate this spread as a continuous smear --- a \emph{distribution} of charges.
(See Figure 4-1.)
Let $\rho (x,y,z)$ be the \emph{charge density} at any point $(x,y,z)$ within the distribution,
the amount of charge per unit volume.
In mks, this would be coulombs per meter cubed.
In Figure 4-1, $\rho (2)$ is charge density at the point $(x_2,y_2,z_2)$.
It is approximately $\Delta q_2 / \Delta V_2$, some small amount of charge
per unit volume centered at point (2).  Therefore
%% Eq. (4.15)
\begin{equation}
  \Delta q_2 = \rho (2) \; \Delta V_2
\end{equation}

Now we take Eq. (4.13), replace the sum with an integral, 
move the constant $1 / 4 \pi \epsilon_0$ outside the integral,
replace $q_j$ with $\rho(2) dV_2$, and $r_{1j}$ with $r_{12}$.
Note that point (2) no longer represents a single point, but
now represents any point $(x_2,y_2,z_2)$ within the distribution.
%% Eq. (4.16)
\begin{equation}
  \bv{E}(1) = \frac{1}{4\pi\epsilon_0}
  \int\limits_{\genfrac{}{}{0pt}{1}{\mathrm{all}}{\mathrm{space}}}
  \frac{\rho(2) \; \bv{e}_{12}}{r^2_{12}} \; dV_2
\end{equation}

Replacing $\bv{e}_{12}$ with $\bv{r}_{12} / r_{12}$ we get
%% Eq. (4.17)
\begin{equation}
  \bv{E}(1) = \frac{1}{4\pi\epsilon_0}
  \int\limits_{\genfrac{}{}{0pt}{1}{\mathrm{all}}{\mathrm{space}}}
  \frac{\rho(2) \; \bv{r}_{12}}{r^3_{12}} \; dV_2
\end{equation}

This integral must be written out in explicit detail in order to calculate things.
Again, $\bv{E}(1)$ has three components.
%% Eq. (4.18)
\begin{eqnarray}
  E_x (x_1,y_1,z_1) \! \! \! \! &=& \! \! \! \! \! \! \! \!
  \int\limits_{\genfrac{}{}{0pt}{1}{\mathrm{all}}{\mathrm{space}}} \! \! \! \!
  \frac{\rho(x_2,y_2,z_2) \; (x_1 - x_2) \; dx_2 \; dy_2 \; dz_2}
       {4\pi\epsilon_0 [(x_1 - x_2)^2 + (y_1 - y_2)^2 + (z_1 - z_2)^2]^{3/2}}
       \\
  E_y (x_1,y_1,z_1) \! \! \! \! &=& \! \! \! \! \! \! \! \!
  \int\limits_{\genfrac{}{}{0pt}{1}{\mathrm{all}}{\mathrm{space}}} \! \! \! \!
  \frac{\rho(x_2,y_2,z_2) \; (y_1 - y_2) \; dx_2 \; dy_2 \; dz_2}
       {4\pi\epsilon_0 [(x_1 - x_2)^2 + (y_1 - y_2)^2 + (z_1 - z_2)^2]^{3/2}}
       \nonumber\\
  E_z (x_1,y_1,z_1) \! \! \! \! &=& \! \! \! \! \! \! \! \!
  \int\limits_{\genfrac{}{}{0pt}{1}{\mathrm{all}}{\mathrm{space}}} \! \! \! \!
  \frac{\rho(x_2,y_2,z_2) \; (z_1 - z_2) \; dx_2 \; dy_2 \; dz_2}
       {4\pi\epsilon_0 [(x_1 - x_2)^2 + (y_1 - y_2)^2 + (z_1 - z_2)^2]^{3/2}}
       \nonumber
\end{eqnarray}

With this formula we can calculate the fields around any set of electrostatic charges.
We can write software to implement the formula and describe a charge distribution,
and then let a computer calculate the fields.
However, sometimes it's easier to calculate the fields through clever guesswork,
by being \emph{smart}.

