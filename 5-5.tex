%% 5-5.tex
\section{The field of a line change}

We wish to know the electric field produced by electric charges
along a straight line of infinite length.
The charges are uniformly distributed.
Let $\lambda$ represent the charge per unit length.

Using arguments of symmetry, we assume the electric field is radial and outward 
at every point along the line. We also assume the electric field has the same 
magnitude at all points equidistant from the line.

Figure 5-5 shows the \textsc{line charge} surrounded by imaginary \textsc{gaussian surface}.
This surface is a cylinder whose axis is the line charge.
The length of the cylinder is 1.  The radius is $r$.
The normal component $E$ is the magnitude of the electric field at any point 
on the cylinder's surface.

Gauss's law says that the total flux from the surface is equal to 
the charge inside divided by $\epsilon_0$.
The total flux is $E$ times the surface area of the cylinder, or $E \cdot 2 \pi r$.
The charge inside is $\lambda$ because the length of the cylinder is 1.
%% nonumber
\begin{equation*}
  E \cdot 2 \pi r = \lambda / \epsilon_0
\end{equation*}

We divide both sides by $2 \pi r$.
This gives us an equation for the electric field of a line charge.
Note that $E$ is depends inversely on $r$ rather than $r^2$ or $r^3$.
%% Eq. (5.2)
\begin{equation}
  E = \frac{\lambda}{2 \pi \epsilon_0 r}
\end{equation}
