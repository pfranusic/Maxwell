%% 2-5.tex
\section{Operations with $\grad$}

We can compute the dot product of $\grad$ with the vector field $\bv{h}$.
We use equation (2.7), the sum of three products.
Remember that $\grad = (\nabla_x, \nabla_y, \nabla_z)$ and $\bv{h} = (h_x,h_y,h_z)$.
% Eq. (2.32)
\begin{equation}
  \grad \cdot \bv{h} = \nabla_x h_x + \nabla_y h_y + \nabla_z h_z
\end{equation}

If we replace the grad terms $(\nabla_x, \nabla_y, \nabla_z)$ with
the hungry partials in equation (2.29) we get
% Eq. (2.33)
\begin{equation}
  \grad \cdot \bv{h} = \dfdx{h_x}{x} + \dfdx{h_y}{y} + \dfdx{h_z}{z}
\end{equation}

We can use a different coordinate system.
Instead of $\grad$ we use $\grad'$.  This gives us
% Eq. (2.34)
\begin{equation}
  \grad' \cdot \bv{h} = \dfdx{h_{x'}}{x'} + \dfdx{h_{y'}}{y'} + \dfdx{h_{z'}}{z'}
\end{equation}

Even though we have two different equations for two different coordinate systems,
the two products are equal, because the scalar product of two vectors is invariant 
under a coordinate transformation.
The vector field $\bv{h}$ is the same in both systems.
The vector operators $\grad$ and $\grad'$ are also the same ---
they just have different variable names, depending on the coordinate system.
% Eq. (2.35)
\begin{equation}
  \grad' \cdot \bv{h} = \grad \cdot \bv{h}
\end{equation}

The dot product of $\grad$ with any vector is called the \emph{divergence}.
% Eq. (2.36)
\begin{equation}
  \grad \cdot \bv{h} = \mathrm{div\ } \bv{h}
\end{equation}

The cross product of $\grad$ with any vector is called the \emph{curl}.
% Eq. (2.37)
\begin{equation}
  \grad \times \bv{h} = \mathrm{curl\ } \bv{h}
\end{equation}

The curl is a vector with three components.
The calculation of each component is specified in equation (2.2)
and given again here, but with the partial terms.
% Eq. (2.38)
\begin{equation}
  \big( \grad \times \bv{h} \big)_z
  = \nabla_x h_y - \nabla_y h_x
  = \dfdx{h_y}{x} - \dfdx{h_x}{y}
\end{equation}
% Eq. (2.39)
\begin{equation}
  \big( \grad \times \bv{h} \big)_x
  = \nabla_y h_z - \nabla_z h_y
  = \dfdx{h_z}{y} - \dfdx{h_y}{z}
\end{equation}
% Eq. (2.40)
\begin{equation}
  \big( \grad \times \bv{h} \big)_y
  = \nabla_z h_x - \nabla_x h_z
  = \dfdx{h_x}{z} - \dfdx{h_z}{x}
\end{equation}

The vector operator $\grad$ allows us to write Maxwell's equations in a succinct form.
In the first equation, $\rho$ (rho) represents the \emph{electric charge density},
the amount of charge per unit volume.
In the fourth equation, $\bv{j}$ represents the \emph{electric current density},
the rate at which charge flows through a unit area per second.
Of course, $c$ represents the speed of light, and $\epsilon_0$ is a convenient constant.
% Eq. (2.41)
\begin{center}
  \emph{Maxwell's\ Equations}
\end{center}
\begin{eqnarray}
  \grad \cdot \bv{E}       &=&  \frac{\rho}{\epsilon_0} \nonumber \\
  \nonumber \\
  \grad \times \bv{E}      &=&  - \dfdx{\bv{B}}{t} \\
  \nonumber \\
  \grad \cdot \bv{B}       &=&  0 \nonumber \\
  \nonumber \\
  c^2 \grad \times \bv{B}  &=&  \dfdx{\bv{E}}{t} + \frac{\bv{j}}{\epsilon_0} \nonumber
\end{eqnarray}



\begin{comment}
%%%%%%%%%%%%%%%%%%%%%%%%%%%%%%%%%%%%%%%%%%%%%%%%%%%%%%%%%%%%%%%%%%%%%%%%%%%%%%%%

\begin{equation*}
  \grad = \Big( \nabla_x, \nabla_y, \nabla_z \Big)
  = \Bigg( \dfdx{}{x}, \dfdx{}{y}, \dfdx{}{z} \Bigg)
\end{equation*}

\begin{equation*}
  \grad' = \Big( \nabla'_x, \nabla'_y, \nabla'_z \Big)
  = \Bigg( \dfdx{}{x'}, \dfdx{}{y'}, \dfdx{}{z'} \Bigg)
\end{equation*}

\begin{equation*}
  \grad \times \bv{h} = \Big(
  \big( \grad \times \bv{h} \big)_x,
  \big( \grad \times \bv{h} \big)_y,
  \big( \grad \times \bv{h} \big)_z
  \Big)
\end{equation*}

%%%%%%%%%%%%%%%%%%%%%%%%%%%%%%%%%%%%%%%%%%%%%%%%%%%%%%%%%%%%%%%%%%%%%%%%%%%%%%%%
\end{comment}
