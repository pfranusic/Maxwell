%% 6-8.tex
\section{A point charge near a conducting plane}

%% Eq. (6.28)
\begin{equation}
  E_{n+} = - \frac{1}{4\pi\epsilon_0} \frac{aq}{(a^2 + \rho^2)^{3/2}}
\end{equation}

%% Eq. (6.29)
\begin{equation}
  \sigma(\rho) = \epsilon_0 E (\rho) = - \frac{2aq}{4 \pi (a^2 + \rho^2)^{3/2}}
\end{equation}

%% Eq. (6.30)
\begin{equation}
  F = \frac{1}{4\pi\epsilon_0} \frac{q^2}{(2a)^2}
\end{equation}



Given the situation shown in Figure 6-10,
we would like to know how the negative charges are distributed on the conducting plate.
The charge density $\sigma$ is a function of the distance $\rho$ 
between some point P and the point directly under the positive charge.
We wish to know how the charge density $\sigma(\rho)$ equation (6.29) was derived.
From equation (6.3) we have
\begin{eqnarray}
  \boldsymbol{E} &=& -\nabla\phi \nonumber\\
  &=& \bigg( -\dfdx{\phi}{x} , -\dfdx{\phi}{y} , -\dfdx{\phi}{z} \bigg) \nonumber
\end{eqnarray}

In the figure, we will assume that the positive point charge and the negative image charge
are both sitting on the $x$-axis, and that the right side of the conducting plate is 
sitting on the $yz$-plane.  Point P has coordinates (0,y,z).
We don't know what the $y$ and $z$ coordinates are, but we do know that $y^2+z^2=\rho^2$.
Therefore, the distance $r$ from point P to the positive charge is $\sqrt{x^2+\rho^2}$.
So, from equation (4.23), we have
\begin{eqnarray}
  \phi(x,y,z) &=& \frac{q}{4\pi\epsilon_0} \frac{1}{r} \nonumber\\
  &=& \frac{q}{4\pi\epsilon_0} \frac{1}{\sqrt{x^2+\rho^2}} \nonumber
\end{eqnarray}

In equation (6.28), $E_{n+}$ is the component normal to the surface of the field 
from the \emph{positive} charge.
This component must be one of the partial derivatives of $\phi$.  But which one?
If we let $u=x^2+\rho^2$, then it appears that
\begin{eqnarray}
  E_{n+} &=& -\dfdx{\phi}{x} \nonumber\\
  &=& -\dfdx{}{x} \frac{q}{4\pi\epsilon_0} \frac{1}{\sqrt{x^2+\rho^2}} \nonumber\\
  &=& -\frac{q}{4\pi\epsilon_0} \dfdx{}{u} (u^{-1/2}) \dfdx{}{x} (x^2+\rho^2) \nonumber\\
  &=& -\frac{q}{4\pi\epsilon_0} \bigg(-\frac{1}{2}u^{-3/2}\bigg)\bigg(2x\bigg) \nonumber\\
  &=& +\frac{1}{4\pi\epsilon_0} \frac{xq}{(x^2+\rho^2)^{3/2}} \nonumber
\end{eqnarray}

\textcolor{red}{
If we set $x=a$, then we have equation (6.28), except for the sign.
So what did I do wrong?
}

