%% 1-3.tex
\section{Characteristics of vector fields}

There are two mathematically important properties of a vector field
which we will use in our description of the laws of electricity
from the field point of view.  
These two properties are \emph{flux} and \emph{circulation}.

Imagine a balloon made of porous rubber and inflated with air.
We wish to quantify the outflow of air through its surface.
The ``flux of velocity'' through the surface is 
the net amount of air going out through the surface per unit time.
The ``flux'' through a surface element
is equal to the component of the velocity perpendicular 
to the element times the area of the element.
For any vector field, the \emph{flux} through an arbitrary closed surface
is defined as the average outward normal component of the vector
multiplied by the area of the surface.
\begin{equation}
  \mathrm{Flux} = (\mathrm{avg\ normal\ component}) \cdot (\mathrm{surface\ area})
\end{equation}

Now imagine a turbulent body of cold water.
The water is instantaneously frozen, all but the water inside a flexible hollow hoop
which materializes at the same moment.
The water inside the hoop continues to circulate due to its momentum.
The ``circulation'' is the speed of this water times the circumference of the hoop.
For any vector field, the \emph{circulation} around an arbitrary closed curve
is defined as the average tangential component of the vector
multiplied by the circumference of the curve.
\begin{equation}
  \mathrm{Circ} = (\mathrm{avg\ tangential\ component}) \cdot (\mathrm{circumference})
\end{equation}

We extend these two ideas to electric and magnetic fields.
With just these two ideas --- flux and circulation ---
we can describe all the laws of electricity and magnetism at once.


\begin{comment}
%%%%%%%%%%%%%%%%%%%%%%%%%%%%%%%%%%%%%%%%%%%%%%%%%%%%%%%%%%%%%%%%%%%%%%%%%%%%%%%%

What are units for electric flux? For magnetic flux?
And what about electric circulation and magnetic circulation?
We can derive all of these from the definitions in equations (1.4) and (1.5).

Electric flux is the product of the average normal component 
of the electric field vector and the surface area.
The unit for an electric field is nt/coul.
The unit for surface area is mt$^2$.
The product of these two is the unit for electric flux.
\begin{equation*}
  \left( \frac{nt}{coul} \right) \cdot \left( mt^2 \right)
  = \frac{nt \cdot mt^2}{coul}
  \equiv \mathrm{Electric\ flux}
\end{equation*}

Magnetic flux is the product of the average normal component 
of the magnetic field vector and the surface area.
The unit for a magnetic field is nt$\cdot$sec/coul$\cdot$mt.
Again, the unit for surface area is mt$^2$.
The product of these two is the unit for magnetic flux.
\begin{equation*}
  \left( \frac{nt \cdot sec}{coul \cdot mt} \right) \cdot \left( mt^2 \right)
  = \frac{nt \cdot mt \cdot sec}{coul}
  \equiv \mathrm{Magnetic\ flux}
\end{equation*}

Electric circulation is the product of the average tangential component
of the electric field vector and the circumference.
The unit for an electric field is nt/coul.
The unit for circumference is mt.
The product of these two is the unit for electric circulation.
\begin{equation*}
  \left( \frac{nt}{coul} \right) \cdot \left( mt \right)
  = \frac{nt \cdot mt}{coul}
  \equiv \mathrm{Electric\ circ.}
\end{equation*}

Magnetic circulation is the product of the average tangential component
of the magnetic field vector and the circumference.
The unit for a magnetic field is nt$\cdot$sec/coul$\cdot$mt.
Again, the unit for circumference is mt.
The product of these two is the unit for magnetic circulation.
\begin{equation*}
  \left( \frac{nt \cdot sec}{coul \cdot mt} \right) \cdot \left( mt \right)
  = \frac{nt \cdot sec}{coul}
  \equiv \mathrm{Magnetic\ circ.}
\end{equation*}

%%%%%%%%%%%%%%%%%%%%%%%%%%%%%%%%%%%%%%%%%%%%%%%%%%%%%%%%%%%%%%%%%%%%%%%%%%%%%%%%
\end{comment}

