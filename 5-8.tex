%% 5-8.tex
\section{Is the field of a point charge exactly $\boldsymbol{1/r^2}$?}

``Gauss' law is true only because the coulomb force
depends exactly on the square of the distance.''
Figure 5-9 shows a large circle that represents a spherical shell of charge.
A point $P$ is somewhere inside the sphere, not necessarily at the center.
Two right circular cones share the vertex $P$.
Each cone intersects the sphere where it describes a circle.
The areas of the circles are $\Delta a_1$ and $\Delta a_2$.

The two cones are symmetrical, such that every line that intersects one circle 
and the vertex $P$, intersects the other circle.
Two right triangles (not shown) may be defined, one for each cone.
Each triangle has vertices $P$, the center of a circle, and the edge of a circle.
Let $r_1$ and $r_2$ be the lengths of the hypotenuses.
Let $b_1$ and $b_2$ (not shown) be the lengths of the radii of the two circles.
Let $\theta$ be the half angle of the cones at the vertex $P$.
We then have $b_1 = r_1 \sin \theta$ and $b_2 =r_2 \sin \theta$,
and the areas of the two circles are
\begin{eqnarray*}
  \Delta a_1 &=& \pi (r_1 \sin \theta)^2 \\
  \Delta a_2 &=& \pi (r_2 \sin \theta)^2
\end{eqnarray*}
The ratio of the two areas, after cancellation of $\pi \sin^2 \theta$ 
in the numerator and the denominator, is equal to the ratio of the two square hypotenuses.
%% nonumber
\begin{equation*}
  \frac{\Delta a_2}{\Delta a_1} = \frac{r^2_2}{r^2_1}
\end{equation*}

The charge on the sphere is uniformly distributed.
Let $\sigma$ be the charge per unit area.
The charge inside the first circle is $q_1 = \sigma \Delta a_1$.
The charge inside the second circle is $q_2 = \sigma \Delta a_2$.
After cancellation of $\sigma$, the ratio of the two charges
is equal to the ratio of the two areas.
%% nonumber
\begin{equation*}
  \frac{q_2}{q_1} = \frac{\Delta a_2}{\Delta a_1} 
\end{equation*}

When we apply the transitive property of equality to the two previous equations,
we see that the ratio of the two charges is equal to the ratio of the two square hypotenuses,
or $q_2/q_1 = r_2^2 / r_1^2$. Rearranging, we get
\begin{equation*}
  \frac{q_2/r_2^2}{q_1/r_1^2} = 1
\end{equation*}

Coulomb's law says that for some point $P$ and some static charge $q$
separated by some distance $r$, the magnitude of the electric field at $P$ 
is given by the equation $E = -q / 4\pi\epsilon_0 r^2$.
Let $E_1$ and $E_2$ be the magnitudes of the two fields produced by $q_1$ and $q_2$
at the common vertex $P$ in Figure 5-9.
\begin{equation*}
  E_1 = \frac{-q_1}{4 \pi \epsilon_0 r_1^2} \qquad
  E_2 = \frac{-q_2}{4 \pi \epsilon_0 r_2^2}
\end{equation*}

After cancellation of $-4\pi\epsilon_0$, the ratio of the two magnitudes is
%% nonumber
\begin{equation*}
  \frac{E_2}{E_1} = \frac{q_2/r_2^2}{q_1/r_1^2}
\end{equation*}

Again, by the transitive property of equality, we have $E_2/E_1 = 1$.
This means that $E_2 = E_1$. In other words, these two fields cancel each other out.
We can then extend this argument for the entire surface of the sphere
to demonstrate that the total field at $P$ is zero.

