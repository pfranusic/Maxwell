%% 4-8.tex
\section{Field lines; equipotential surfaces}

We wish to duplicate the drawing in Figure 4-13.
It shows field lines and equipotentials for two equal and opposite point charges.
We use the fact that $\boldsymbol{E} = -\nabla\phi$.
We first calculate the scalar potential field $\phi$ 
and then the vector electric field $\boldsymbol{E}$.
We simplify our formula for $\phi$ by letting $q_p = +1$ and $q_n = -1$.
We also let $k = (4\pi\epsilon_0)^{-1}$.
We compute $r_p$ and $r_n$ separately.
\begin{eqnarray}
  r_p &=& \sqrt{(x-x_p)^2 + (y-y_p)^2 + (z-z_p)^2} \nonumber\\
  r_n &=& \sqrt{(x-x_n)^2 + (y-y_n)^2 + (z-z_n)^2} \nonumber\\
  \phi(x,y,z) &=& k \bigg( \frac{1}{r_p} - \frac{1}{r_n} \bigg) \nonumber
\end{eqnarray}

The program \textsf{point-charges.c} has been partially written and debugged.
To run the program, type ``\textsf{./point-charges}'' on the command line with no arguments.
A Cocoa window is slowly filled line by line.
To stop the program, kill the window.

The program has a number of deficiencies.
It is written to a Cocoa window rather than a file.
It displays the equipotentials in solid red on a green background,
rather than black dotted lines on a white background.
The field lines are not displayed at all.
The program obviously needs more work.

