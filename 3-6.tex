%% 3-6.tex
\section{The circulation around a square; \\ Stokes' theorem}

We wish to derive an equation for the circulation around a square.
We'll make some assumptions in order to simplify the derivation.
The surface in Figure 3-9 may not be flat,
but each square is so small that we can assume it to be flat.
We'll also assume that our square lies in a plane which is
orthogonal to one of the Cartesian axes, say, the $xy$ plane.
Finally, we'll assume that the sides of our square
are parallel to the $x$ and $y$ axes.

Recall that a vector field $\bv{C}$ can be thought of as
the gradient of a scalar field, e.g. temperature $T$.
Each component of the vector specifies a rate of change
in one of three directions. In the case of temperature,
the units for each component might be degrees per centimeter.
\[ \bv{C} = \grad T = \Bigg( \dfdx{T}{x}, \dfdx{T}{y}, \dfdx{T}{z} \Bigg) \]

Consider the square in Figure 3-10.
The circulation is expressed as the sum of four terms.
Each term is the product of a tangential component of $\bv{C}$ and the length of a side.
%% Eq. (3.31)
\begin{equation}
  \oint_\Gamma \bv{C} \cdot d\bv{s} =
  C_x(1) \Delta x + C_y(2) \Delta y - C_x(3) \Delta x - C_y(4) \Delta y
\end{equation}

We combine the two terms that contain $\Delta x$.
%% Eq. (3.32)
\begin{equation}
  [C_x(1) - C_x(3)] \; \Delta x
\end{equation}

The distance between side 1 and side 3 is small,
and therefore the difference between $C_x(1)$ and $C_x(3)$ is small.  But it's not zero.
The two sides are separated in the $y$ direction by $\Delta y$.
The rate of change between the two $C_x$ values is $\partial C_x / \partial y$.
Therefore we have
%% Eq. (3.33)
\begin{equation}
  C_x(3) = C_x(1) + \dfdx{C_x}{y} \; \Delta y
\end{equation}

We replace the two $\Delta x$ terms in equation (3.31) with this difference.
%% Eq. (3.34)
\begin{equation}
  [C_x(1) - C_x(3)] \Delta x = - \dfdx{C_x}{y} \; \Delta x \; \Delta y
\end{equation}

We apply the same logic to replace the two $\Delta y$ terms.
%% Eq. (3.35)
\begin{equation}
  C_y(2) \Delta y - C_y(4) \Delta y = \dfdx{C_y}{x} \; \Delta x \; \Delta y
\end{equation}

This gives us the product of a difference and an area.
%% Eq. (3.36)
\begin{equation}
  \bigg( \dfdx{C_y}{x} - \dfdx{C_x}{y} \bigg) \; \Delta x \; \Delta y
\end{equation}

The difference term just happens to be the $z$ component of 
the cross-product $(\grad \times \bv{C})$.
And we can let area $\Delta a = \Delta x \Delta y$.
This gives us 
\[ (\grad \times \bv{C})_z \Delta a \]

Now we'll eliminate the Cartesian coordinates altogether.
In Figure 3-10, the $z$-axis is normal to our square in the $xy$ plane.
But if our square were not in the $xy$ plane, it would still have a normal axis $n$.
So we can replace the $z$ and simply use $n$ instead.
This gives us 
\[ (\grad \times \bv{C})_n \Delta a \]

We replace the four terms with the single cross-product term.
%% Eq. (3.37)
\begin{equation}
  \oint \bv{C} \cdot d\bv{s}
  = (\grad \times \bv{C})_n \; \Delta a 
  = (\grad \times \bv{C}) \cdot \bv{n} \; \Delta a
\end{equation}

Finally, we take the closed loop $\Gamma$, fill it in with some surface,
subdivide that surface into an infinite number of infinitesimal squares,
compute the circulation around each square, and then take the sum.
%% Eq. (3.38)

\hspace{2em} \textsc{Stokes' Theorem:}
\begin{equation}
  \oint_\Gamma \bv{C} \cdot d\bv{s} = \int_S (\grad \times \bv{C})_n \; da
\end{equation}
\hspace{2em} where $S$ is any surface bounded by $\Gamma$.

\newpage

