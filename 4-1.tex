%% 4-1.tex
\section{Statics}

The previous three chapters laid the mathematical foundation
for the study of electromagnetism, all of which is contained 
in the four Maxwell equations.

%% Eq. (4.1)
\begin{equation}
  \grad \cdot \bv{E} = \frac{\rho}{\epsilon_0}
\end{equation}

%% Eq. (4.2)
\begin{equation}
  \grad \times \bv{E} = - \dfdx{\bv{B}}{t}
\end{equation}

%% Eq. (4.3)
\begin{equation}
  c^2 \grad \times \bv{B} = \dfdx{\bv{E}}{t} + \frac{\bv{j}}{\epsilon_0}
\end{equation}

%% Eq. (4.4)
\begin{equation}
  \grad \cdot \bv{B} = 0
\end{equation}

\vspace{2em}
We will first consider situations where nothing depends on time.
These are where all charges are fixed in space.
In these situations, all time derivatives in the Maxwell equations are zero.
This leaves two equations with only $\bv{E}$ terms
and two equations with only $\bv{B}$ terms.
There are no equations with both $\bv{E}$ and $\bv{B}$ terms.
This means that \emph{when all charges and currents are static,
electriciy and magnetism are distinct phenomena.}

\emph{Electrostatics:} 

%% Eq. (4.5)
\begin{equation}
  \grad \cdot \bv{E} = \frac{\rho}{\epsilon_0}
\end{equation}

%% Eq. (4.6)
\begin{equation}
  \grad \times \bv{E} = 0
\end{equation}

\emph{Magnetostatics:} 

%% Eq. (4.7)
\begin{equation}
  \grad \times \bv{B} = \frac{\bv{j}}{\epsilon_0 c^2}
\end{equation}

%% Eq. (4.8)
\begin{equation}
  \grad \cdot \bv{B} = 0
\end{equation}

Electrostatics and magnetostatics are ideal for learning about 
the mathematical properties of vector fields.
Electrostatics is a vector field with fixed divergence (4.5) and zero curl (4.6).
Magnetostatics is a vector field with fixed curl (4.7) and zero divergence (4.8).

