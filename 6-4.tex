%% 6-4.tex
\section{The dipole potential as a gradient}

We wish to demonstrate that
\[ \grad \left(\frac{1}{r}\right) = -\frac{\bv{r}}{r^3} \]

We begin with vector $\bv{r}=(x,y,z)$ and magnitude $r=\sqrt{x^2+y^2+z^2}$.
To simply things, we let $u=r^2$.  Then we have
\[ u^{-1/2} = \frac{1}{r} \qquad u^{-3/2} = \frac{1}{r^3} \]

And since $u=x^2+y^2+z^2$ we have the three partial derivatives
\[ \dfdx{u}{x}=2x \qquad \dfdx{u}{y}=2y \qquad \dfdx{u}{z}=2z \]

Now we're ready to calculate the gradient of $1/r$.
We expand the gradient vector into its three components,
replace each $1/r$ with $u^{-1/2}$,
and apply the chain rule to each component.
\begin{eqnarray*}
  \grad \left(\frac{1}{r}\right) 
  &=& \left( \dfdx{}{x}\!\left(\frac{1}{r}\right),\quad
             \dfdx{}{y}\!\left(\frac{1}{r}\right),\quad
	     \dfdx{}{z}\!\left(\frac{1}{r}\right) \right) \\
  &=& \left( \dfdx{}{x}u^{-1/2},\quad
	     \dfdx{}{y}u^{-1/2},\quad
	     \dfdx{}{z}u^{-1/2} \right) \\
  &=& \left( \dfdx{}{u}u^{-1/2}\;\dfdx{u}{x} \;,\;
             \dfdx{}{u}u^{-1/2}\;\dfdx{u}{y} \;,\;
             \dfdx{}{u}u^{-1/2}\;\dfdx{u}{z} \right)
\end{eqnarray*}

We take the derivatives, replace each $u^{-3/2}$ with $1/r^3$,
and collapse the three components into the vector $\bv{r}$
times the constant $-1/r^3$.
\begin{eqnarray*}
  &=& \left( -\frac{1}{2}u^{-3/2}\;2x,\quad
	     -\frac{1}{2}u^{-3/2}\;2y,\quad
	     -\frac{1}{2}u^{-3/2}\;2z \right) \\
  &=& \left( -\frac{1}{r^3}x,\quad -\frac{1}{r^3}y,\quad -\frac{1}{r^3}z \right) \\
  &=& -\frac{1}{r^3} \bigg(x,\; y,\; z \bigg) \\
  &=& -\frac{\bv{r}}{r^3} \qquad\Box
\end{eqnarray*}

We can now replace the $\bv{r}/r^3$ term in Equation (6.13) 
and write the dipole potential in a new form.
%% Eq. (6.16)
\begin{equation}
  \phi = - \frac{1}{4\pi\epsilon_0} \bv{p} \cdot \grad \left( \frac{1}{r} \right)
\end{equation}

Remember that $\bv{e}_r=\bv{r}/r$ is the unit radial vector
where $\bv{r}=(x,y,z)$ and $r=\sqrt{x^2+y^2+z^2}$. Then
%% nonumber
\begin{equation*}
  \grad \left( \frac{1}{r} \right) = - \frac{\bv{r}}{r^3}
  = - \frac{\bv{e}_r}{r^2}
\end{equation*}

The dipole potential $\phi$ can also be derived using the superposition principle.
Simply compute the two potentials $\phi_+$ and $\phi_-$ and then add them together.
We begin with the potential $\phi_0$ at the point $P$ at $(x,y,z)$ 
for a charge that resides at the orgin.
We use Equation (4.23) but omit the $1/4\pi\epsilon_0$ term for the sake of convenience.
%% nonumber
\begin{equation*}
  \phi_0 = \frac{q}{r}
\end{equation*}

We need the potential $\phi_+$ at point $P$ for the positive charge $+q$
that resides $\Delta z$ above the orgin.
%% Eq. (6.17)
\begin{eqnarray}
  \phi_+ &=& \phi_0 \; + \; \Delta \phi_{+} \nonumber\\
  &=& \left[\frac{+q}{r}\right] + \left[\dfdx{\phi_0}{z}\;(-\Delta z)\right] \nonumber\\
  &=& \frac{+q}{r} - \dfdx{}{z}\left(\frac{+q}{r}\right) \frac{d}{2}
\end{eqnarray}

We also need the potential $\phi_-$ at point $P$ for the negative charge $-q$
that resides $\Delta z$ below the orgin.
%% Eq. (6.18)
\begin{eqnarray}
  \phi_- &=& \phi_0 \; + \; \Delta \phi_{-} \nonumber\\
  &=& \left[\frac{-q}{r}\right] + \left[\dfdx{\phi_0}{z}\;(+\Delta z)\right] \nonumber\\
  &=& \frac{-q}{r} + \dfdx{}{z}\left(\frac{-q}{r}\right) \frac{d}{2}
\end{eqnarray}

Finally we add the two potentials.
The two $q/r$ terms cancel each other, the two $-\partial/\partial z$ terms add together,
and we move the $q$ outside the parentheses.
%% Eq. (6.19)
\begin{eqnarray}
  \phi &=& \phi_{+} \; + \; \phi_{-} \nonumber\\
  &=& \left[ \frac{+q}{r} - \dfdx{}{z}\left(\frac{+q}{r}\right) \frac{d}{2} \right]
    + \left[ \frac{-q}{r} + \dfdx{}{z}\left(\frac{-q}{r}\right) \frac{d}{2} \right] \nonumber\\
  &=& - \dfdx{}{z} \left(\frac{1}{r}\right) qd
\end{eqnarray}

To get Equation (6.19) we oriented the dipole on the $z$-axis,
so that the distance from $P$ was simply $\Delta z$.
But in the general case, we use $\Delta\bv{r}$ as the distance,
and the formulas for $\Delta\phi_+$ and $\Delta\phi_-$ are scalar products
to take care of all three dimensions.
We replace $\Delta\bv{r}$ with $\bv{d}/2$.
%% nonumber
\begin{eqnarray*}
  \Delta\phi_+ &=& \grad\left(\frac{+q}{r}\right)\cdot\left(-\Delta\bv{r}\right) 
  \quad = \quad -\grad\left(\frac{q}{r}\right)\cdot\frac{\bv{d}}{2} \\
  \Delta\phi_- &=& \grad\left(\frac{-q}{r}\right)\cdot\left(+\Delta\bv{r}\right)
  \quad = \quad -\grad\left(\frac{q}{r}\right)\cdot\frac{\bv{d}}{2}
\end{eqnarray*}

The equations for $\phi_+$ and $\phi_-$ then become
%% nonumber
\begin{eqnarray*}
  \phi_+ &=& \phi_0 \; + \; \Delta\phi_+ \\
         &=& \frac{+q}{r} - \grad\left(\frac{q}{r}\right) \cdot \frac{\bv{d}}{2} \\
  \phi_- &=& \phi_0 \; + \; \Delta\phi_- \\
         &=& \frac{-q}{r} - \grad\left(\frac{q}{r}\right) \cdot \frac{\bv{d}}{2} \\
\end{eqnarray*}

We add $\phi_+$ and $\phi_-$ to get $\phi$.
As before, the $q/r$ terms cancel, the partial derivative terms add together, 
and we move the $q$ outside the parentheses.
We also put back the $1/4\pi\epsilon_0$ term, which then multiplies each of the three
vector components.
%% nonumber
\begin{eqnarray*}
  \phi &=& -\frac{1}{4\pi\epsilon_0}\grad\left(\frac{1}{r}\right) \cdot q \bv{d} \\
       &=& -\grad \left(\frac{1}{4\pi\epsilon_0 r}\right) \cdot q \bv{d}
\end{eqnarray*}

Let $\bv{p}=q\bv{d}$ and
unit point charge potential $\Phi_0=1/4\pi\epsilon_0 r$. Then
%% Eq. (6.20)
\begin{equation}
  \phi\; = \; -\grad\Phi_0 \cdot \bv{p} 
\end{equation}

\textcolor{red}{The last three equations in this section are not numbered.
They are used to make the point that a complex problem can be solved
using the superposition principle.
I don't know how these three equations are derived.
I'll punt for now and return to them later.}

%% nonumber
\begin{equation*}
  \sigma = \sigma_0 \cos \theta
\end{equation*}

%% nonumber
\begin{equation*}
  p = \frac{4\pi\sigma_0 a^3}{3}
\end{equation*}

%% nonumber
\begin{equation*}
  E = \frac{\sigma_0}{3\epsilon_0}
\end{equation*}

