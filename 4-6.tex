%% 4-6.tex
\section{Gauss's law; divergence of $\bv{E}$}

Remember that $E_n$ is the normal component of $\bv{E}$
at some point in the ``middle'' of some infinitesimal area $da$
on the surface of $S$,
where $\bv{E}$ is the electric field produced by all charges,
and $\bv{n}$ is the unit vector pointing outward at right angles to the surface.

Suppose instead of one charge $q$ we have two charges $q_1$ and $q_2$.
Then $\bv{E}_1$ represents the electric field produced by $q_1$ alone,
and $\bv{E}_2$ represents the electric field produced by $q_2$ alone.
Because of the principle of superposition for electric fields, 
we can say that $\bv{E} = \bv{E}_1 + \bv{E}_2$.

Now we can take the left side of Eq. (4.32) and replace the $E_n$ inside the integral
with the sum of the two normal components of $\bv{E}_1$ and $\bv{E}_2$.
That is, $E_{1n}$ and  $E_{2n}$.
By the rules of calculus we can split the single integral into two separate integrals.
%% Eq. (4.33)
\begin{equation}
  \int_S \left( E_{1n} + E_{2n} \right) \; da =
  \int_S E_{1n} \; da + \int_S E_{2n} \; da 
\end{equation}

Each of the integrals on the right is either 0 or some charge $q_i$ divided by $\epsilon_0$.
This gives rise to Gauss' law for the flux of an electric field produced by one or more charges.
We replace the right side of Eq. (4.32) with a term that reflects the superposition principle.

\hspace{2em}\emph{Gauss' law:}
\vspace{-1em}
%% Eq. (4.34)
\begin{equation}
  \int\limits_{\genfrac{}{}{0pt}{1}{\mathrm{any\ closed}}{\mathrm{surface\ }S}}
  E_n \; da = \frac{\mathrm{sum\ of\ charges\ inside}}{\epsilon_0}
\end{equation}

The ``sum of charges inside'' can be represented by $Q_{\mathrm{int}}$.
And the normal component $E_n$ is simply the dot product of 
the two vectors $\bv{E}$ and $\bv{n}$.
%% Eq. (4.35)
\begin{equation}
  \int\limits_{\genfrac{}{}{0pt}{1}{\mathrm{any\ closed}}{\mathrm{surface\ }S}}
  \bv{E} \cdot \bv{n} \; da = \frac{Q_{\mathrm{int}}}{\epsilon_0}
\end{equation}

For a discrete number of charges, $Q_{\mathrm{int}}$ is the 
sum of all charges inside $S$.
%% Eq. (4.36)
\begin{equation}
  Q_{\mathrm{int}} = \sum_{\mathrm{inside\ }S} q_i
\end{equation}

If we want to think of the charges as evenly distributed in the volume within the surface,
then we need to use the idea of charge density, represented by $\rho$ (charge per volume).
We subdivide the volume $V$ into an infinite number of infinitesimal volumes $dV$.
The charge of each volume is $\rho \; dV$.
Since there are an infinite number of charges, we need to integrate to get the total charge.
%% Eq. (4.37)
\begin{equation}
  Q_{\mathrm{int}} = 
  \int\limits_{\genfrac{}{}{0pt}{1}{\mathrm{volume}}{\mathrm{inside\ }S}}
  \rho \; dV
\end{equation}

Now we write a differential form of Gauss' law.
We start with Eq. (3.17), replace vector $\bv{C}$ with vector $\bv{E}$,
and replace $\Delta V$ with $dV$.  This gives us
%% nonumber
\begin{equation*}
  \int \bv{E} \cdot \bv{n} = \left( \grad \cdot \bv{E} \right) dV
\end{equation*}

The left side of this equation is the same as the left side of Eq. (4.35).
So we can set both right sides equal to each other.
We can also replace the $Q_{\mathrm{int}}$ with $\rho \; dV$, 
since $dV$ is an infinitesimal volume.
%% nonumber
\begin{equation*}
  \left( \grad \cdot \bv{E} \right) dV = \frac{\rho \; dV}{\epsilon_0}
\end{equation*}

Finally, we simplify by dividing both sides by $dV$.
This is the differential form of Gauss' law.
It is also one of Maxwell's equations.
%% Eq. (4.38)
\begin{equation}
  \grad \cdot \bv{E} = \frac{\rho}{\epsilon_0}
\end{equation}

